Modern macroeconomic frameworks aim to inform monetary and fiscal policy decision-makers about preferred policy objectives and ways they can be attained. The aim of the dissertation was to build a small-scale two-country NK DSGE SOE model for Scotland and the rest of the UK that would complement the informal dimension of the decision-making process. 

More specifically, the first section provided a brief introduction to Keynesian and Classical schools of thought, and how over time they lead to the development of modern macroeconomic frameworks, actively employed by international organisations and central banks today. The second section presented the DSGE model: households, the government, firms, and equilibrium (market clearing) conditions. In the interest of space, most of the derivations were placed in the Appendix but the section explained the modelled behaviour of endogenous variables. The presentation of the model was followed by a section describing the processes of estimating and calibrating model parameters. Using more than twenty years of data from the Quarterly National Accounts (of Scotland), the estimated model parameters in Scotland were considerably different from their counterparts in the rest of the UK. Steady-state ratios in both economies were calculated using the 1998Q1-2021Q4 sample data.

While the two economies were modelled as symmetrical, the differences in model parameter values allowed the identification of asymmetric dynamic responses to shocks in government spending. The responses were discussed in the Dynamic Responses section. The section analysed impulse response functions and provided an intuition behind the source and direction of the response. Across all policy scenarios, the asymmetric parameter values did not alter the direction of dynamic responses. In fact, the parameter sensitivity analysis showed that only the government spending persistence parameter $\rho_g$ can alter the effect's direction and only when extremely small values are considered. Overall, the analysis of dynamic responses showed that in cases like Scotland, a higher government spending-to-output steady state ratio is linked to increased sensitivity of output, employment, and inflation to changes in government spending. At the same time, consumption showed nearly no variation with the ratio, and real wage exhibited lower sensitivity. The findings also suggest that the introduction of the labour tax softens the responses, given a sudden increase in government spending. Being in or out of a fiscal union made no difference to the rest of the UK, but it did have effect on responses in Scotland: the dynamic responses were ``pulled'' towards those of the rest of the UK. The parameter sensitivity analysis suggests that only extreme values of government spending persistence parameters has considereably influence on dynamic responses. The fiscal feedback parameters influenced only the amount of bonds issued and had limited effect on dynamic responses of other endogenous variables. Finally, the limitations sectioned discussed ways in which the model can be extended to allow more sophisticated policy analysis or improved data fit.