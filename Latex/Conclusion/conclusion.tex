Modern macroeconomic frameworks aim to inform monetary and fiscal policy decision-makers about preferred policy objectives and ways they can be attained. The aim of the dissertation was to build a small-scale two-country NK DSGE SOE model for Scotland and the rest of the UK that would complement the informal dimension of the decision-making process. 

More specifically, the first section provided a brief introduction to Keynesian and Classical schools of thought and how over time, they led to the development of modern macroeconomic frameworks actively employed by international organisations and central banks today. The second section presented the DSGE model: households, the government, firms, and equilibrium (market clearing) conditions. In the interest of space, most of the derivations were placed in the Appendix but the section explained the modelled behaviour of endogenous variables. The presentation of the model was followed by a section describing the processes of estimating and calibrating model parameters. Using more than twenty years of data from the Quarterly National Accounts (of Scotland), the estimated model parameters in Scotland were considerably different from their counterparts in the rest of the UK. Steady-state ratios in both economies were calculated using the 1998Q1-2021Q4 sample data.

While the two economies were modelled as symmetrical, the differences in model parameter values allowed the identification of asymmetric dynamic responses to shocks in government spending. The `Dynamic Responses' section analysed impulse response functions and provided an intuition behind the source and direction of the response. Across all policy scenarios, the asymmetric parameter values did not alter the direction of dynamic responses. In fact, the parameter sensitivity analysis showed that only the government spending persistence parameter $\rho_g$ can alter the effect's direction and only when extremely small values are considered. Overall, the findings suggest that the introduction of the labour tax makes government spending more distortionary. Being in or out of a fiscal union made almost no difference to the rest of the UK, but it significantly affected dynamic responses in Scotland. The parameter sensitivity analysis suggests that only the government spending persistence parameter has a considerable influence on dynamic responses. The fiscal feedback parameters influenced the amount of bonds issued, the labour tax rate, and tax revenue, but the changes had a limited effect on the dynamic responses of other endogenous variables. Finally, the limitations section discussed ways to extend the model to allow more sophisticated policy analysis or improved data fit.