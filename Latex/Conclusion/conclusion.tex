Modern macroeconomic frameworks aim to inform monetary and fiscal policy decision-makers about preferred policy objectives and ways they can be attained. The aim of the dissertation was to build a small-scale two-country NK DSGE SOE model for Scotland and the rest of the UK that would complement the informal dimension of the decision-making process. 

More specifically, the first section overviewed literature on the evolution of macroeconomic frameworks from the 1940s to today. \textcolor{red}{expand}. The second section introduced the model: households, the government, firms, and equilibrium (market clearing) conditions. The section discussed equations key to the characterisation of the equilibrium and presented references to their derivations in the Appendix. It was followed by a section on the application of the model. More specifically, the section discussed the calibration and estimation of the model parameters. Using more than twenty years of data from the Quarterly National Accounts (of Scotland), the estimated model parameters in Scotland were considerably different from their counterparts in the rest of the UK. Steady-state ratios in both economies were calculated using the same sample data. 

While the two economies were modelled as symmetrical, the differences in model parameter values allowed the identification of asymmetric dynamic responses to shocks in government spending. The responses were discussed in the Results section. The section analysed impulse response functions and provided an intuition behind the source and direction of the response with references to corresponding equations from the Model section. Across all policy scenarios, the asymmetric parameter values did not alter the direction of dynamic responses. In fact, the parameter sensitivity analysis showed that only the government spending persistence parameter $\rho_g$ can alter the effect's direction and only when extremely small values are considered. Overall, the analysis of dynamic responses showed that in cases like Scotland, a higher government spending-to-output ratio is linked to increased sensitivity of output, employment, and inflation to changes in government spending. At the same time, consumption showed nearly no variation with the ratio, and real wage exhibited lower sensitivity. Finally, the limitations sectioned discussed ways in which the model can be extended to improve data fit and allow more rigorous analysis.