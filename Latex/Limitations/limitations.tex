The greatest limitation of the model is its scale. For instance, the model assumes that a representative household derives utility from consumption and leisure only. In practice, government spending is not wasteful, and households directly benefit from the existence of public services. However, the government goods (services) did not enter the utility function, which could have enabled a more stimulating analysis. Moreover, the households were assumed to have frictionless access to financial instruments, allowing them to save a part of their consumption. When households are modelled as saving-constrained or poor (\textit{hand-to-mouth}), DSGE results tend to align more with observed empirical data.

In terms of firms, the model made several simplifying assumptions. Firstly, it assumed that only domestic goods' price, not labour (wage), is sticky. In practice, wages are determined not only by labour supply but also by the existence of employment contracts, labour laws, and unions. Secondly, more than labour is needed to produce goods; firms also require private, public, and human capital, which the model ignored in an attempt to retain its small scale. 

\textcite{ricci_2019_essays} model is more tailored to Scotland and the rest of the UK economies. For instance, the government in Westminister administers funding to Scotland in line with the Barnett formula, and there exists oil industry. A more careful consideration is given to the real exchange rate between the two economies to account for a significant price pass-through. The absence of these extensions are limitations of the model employed by this dissertation.