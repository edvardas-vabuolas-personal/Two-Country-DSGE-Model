This section will consider the responses of endogenous variables, given two types of shocks: monetary and fiscal. Although the examination of monetary shocks falls beyond the scope of this dissertation, it serves as a valuable benchmark to assess the accuracy and validity of our model. The Dynare file accompanying this dissertation also allows for deriving dynamic responses to shocks in world output, price level, and preferences ($z_t$). Still, these fall outside the scope of the dissertation and are not discussed. For the monetary policy shock, we use the policy scenario four model, i.e. a single fiscal government that uses labour tax and bond issuance as fiscal instruments to raise budget revenue.

Figure \ref{figure:m_four} presents the responses of fourteen endogenous variables, where two variables, namely the labour tax rate and the interest rate itself, apply across the entire UK. Additionally, ten variables encompass each country's consumption, output, real wage ($wp_t = w_t - p_t$), domestic inflation, and employment (hours worked). The last two variables represent country-specific interest rates, which, although not directly pertinent to the main focus (being included purely for technical reasons), display a co-movement in line with the model's imposition of the monetary union. Generally, dynamic responses are very much in line with textbook literature \parencite[243]{jordigal_2015_monetary}: An increase in interest rate makes saving more attractive, making households incentivised to consume less ``today'' in hopes to consume more ``tomorrow''. A decline in demand for home goods results in a decline in output, which creates an abundance of labour, leading to an increase in unemployment. An increase in interest rate also leads to downward inflationary pressure because goods become relatively less attractive to households compared to saving, and making goods cheaper becomes the best strategy for price-setting, profit-maximising firms. Finally, given that employment decreases, bringing down the tax revenue, the government responds with an increase in the labour tax rate ($\tau^{UK}_t$) to clear the government budget. All effects fade as the economy returns to its steady state equilibrium.
\import{./Graphs}{m_four.tex}

Next, dynamic responses to an increase in government spending are considered. Across all policy scenarios, the responses exhibit consistent behaviour. That is, the government reduces the quantity of available resources to the household, leading to decreased consumption. The negative wealth effect becomes more significant than the substitution effect between consumption and leisure, leading to increased employment. An increase in labour supply allows firms to produce more (output increases) while paying less in real terms (real wage decreases). In all scenarios, the supply of government-issued bonds increases as the government raises funds to cover its spending. In the scenarios where labour tax exists, the government raises the tax rate, as well as issues bonds with the intensity of the two financial instruments determined by the fiscal rule and fiscal policy parameters $\phi_b$ and $\phi_g$. The model considered in this dissertation assumes that the government ``consumes'' domestic goods. Therefore, increasing government spending raises demand for these goods and creates upward inflationary pressure, as shown below in Figures \ref{figure:f_one}-\ref{figure:f_four}. Domestic goods become less competitive internationally, leading to depreciation in terms of trade (displayed only in Figures \ref{figure:f_one} and \ref{figure:f_one_f} in the interest of space). In all IRFs, responses of \textit{domestic} inflation will be displayed as it allows easier comparison with more common closed economy DSGE models found in the literature; however, CPI inflation can be easily inferred from Equation (\ref{eq:cpi_inflation_tot}), which shows that CPI is a sum of domestic inflation and period-to-period changes in terms of trade times the openness parameter $\upsilon$.
\import{./Graphs}{f_one.tex}
\import{./Graphs}{f_one_f.tex}
\import{./Graphs}{f_two.tex}
\import{./Graphs}{f_three.tex}
\import{./Graphs}{f_three_f.tex}
\import{./Graphs}{f_four.tex}
More specifically, under policy scenario four ($G:1,\ \tau:1$), IRFs suggest that a 0.25\% deviation from the steady state government spending in any given quarter results in 0.09\% in Scotland and 0.02\% in the rest of the UK deviation in output in the same quarter.\footnote{Note that the dissertation uses 25 basis points quarterly increase, so the annualised increase would be 1 percentage point. This makes the results more comparable to those found in the literature.} Consumption decreased by 0.4\% and 0.25\%, while employment increased by 0.15\% and 0.05\% in Scotland and the rest of the UK, respectively. A more significant deviation from the steady state of employment level in Scotland also resulted in a more significant decrease in real wages (0.125\%) compared to that of the rest of the UK (0.075\%). Finally, the domestic inflation increase was similar across both countries (0.04\% in Scotland and 0.06\% in the rest of the UK). Tables \ref{table:responses_one_three} and \ref{table:responses_two_four} contrast 200-period cumulative responses between countries for each of the policy scenarios (column ``Diff.'').\footnote{Note that all variables return to their steady-state levels within the first 50 periods. Using 200 as an arbitrarily large number ensures that the full effect is accurately calculated even if extreme persistence parameters were chosen.} It also displays the changes in responses when the labour tax is introduced. In this model, the introduction of labour taxes is found to make government spending more distortionary: the cumulative increase in output and employment across both countries decreased by 0.248\% and 0.372\%, and consumption fell by an additional 0.227\%. These findings apply to both sets of scenarios, 1 \& 3 and 2 \& 4, with slightly larger multipliers observed in the latter pair.

Tables \ref{table:responses_one_two} and \ref{table:responses_three_four} show how the cumulative effect changes when either country joins the fiscal union. When there are no labour taxes, being in or out of a fiscal union makes (almost) no difference for the rest of the UK. However, Scotland is found to be significantly more susceptible to such changes (with or without the existence of labour tax). When Scotland is modelled to join a fiscal union with the rest of the UK, fewer bonds (-0.85\%) are issued, and households face a 0.398\% higher (cumulative) labour tax rate. The result is a more significant decrease in consumption (-1.625\%) and real wage (-0.814\%). Output and employment remain almost unchanged.
\newpage
\import{./Results/Tables}{results_one_three.tex}
\import{./Results/Tables}{results_two_four.tex}
\newpage
\import{./Results/Tables}{results_one_two.tex}
\import{./Results/Tables}{results_three_four.tex}
Moreover, Figures \ref{figure:sensitivity_analysis_phi_b}-\ref{figure:sensitivity_analysis_rho_g} illustrate the sensitivity of dynamic responses given changes in fiscal feedback and persistence parameters $\phi_b, \phi_g, \rho_g$ (and their rUK/UK counterparts). For each figure, one parameter values are allowed to vary while the other two are fixed to prior mean values. The sensitivity analysis is carried out for policy scenario four only. Unsurprisingly, the fiscal policy persistence parameter $\rho_g$ is found to have the most significant impact on the magnitude of responses, with some of the variables even responding in the opposite than expected direction when (extremely) small $\rho_g$ values are considered. According to \textcite[27]{jordigal_2005_understanding}, the persistence parameter is associated with the strength of the wealth effect (the more persistent the government spending shock, the greater the wealth effect is imposed on the households). This is confirmed by Figure \ref{figure:sensitivity_analysis_rho_g}: as $\rho_g$ decreases, the dynamic responses return to their steady-state values quicker. The positive responses of the real wage when $\rho_g \in \{ 0.10, 0.60 \}$ require further attention, but it could be attributed to the price stickiness and flexible wages: output and employment initial responses are dramatically greater to their counterpart responses when $\rho_g = 0.99$, while the initial response of inflation (period-to-period log price level change) is almost unchanged for all $\rho_g$ values. This suggests that nominal wages increase more than sticky prices, resulting in a positive real wage response. As discussed in Section 2: Government, tax revenue and labour tax rate are increasing in $\phi_b$ and $\phi_g$. A high value of either parameter results in a more significant immediate response. However, that does not result in significant changes in responses of other variables as only the amount of bonds issued, the labour tax rate, and the tax revenue are \textit{considerably} sensitive to changes in $\phi_b$ and $\phi_g$.
\import{./Graphs}{sensitivity_analysis_phi_b_test.tex}
\import{./Graphs}{sensitivity_analysis_phi_g_test.tex}
\import{./Graphs}{sensitivity_analysis_rho_g_test.tex}
Finally, regarding the transmission of shock from one country to another, the model does not consider migration, national (UK-wide) wages, and significant price pass-through (low real exchange rate between the two economies). Therefore, in policy scenarios with two governments, endogenous variables in one country do not respond considerably to shocks in government spending in another country; the effect exhibits a magnitude on the order of 1e-13. This is a clear limitation of the model and calls for additional attention. The following section discusses other DSGE extensions that could improve the model's performance.

