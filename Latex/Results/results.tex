This section will consider the responses of endogenous variables, given two types of shocks: monetary and fiscal. Although the examination of monetary shocks falls beyond the scope of this dissertation, it serves as a valuable benchmark to assess the accuracy and validity of our model. The Appendix Z also includes dynamic responses to world output and price level, as well as preferences ($z_t$). This dissertation does not provide a discussion on responses to these shocks. For the monetary policy shock, we use the policy scenario four model, i.e. a single fiscal government that uses labour tax and bond-issuance as fiscal instruments to raise budget revenue.

Figure \textbf{?} presents the responses of fourteen endogenous variables, where two variables, namely the labor tax rate and the interest rate itself, apply across the entire UK. Additionally, ten variables encompass consumption, output, real wage ($wp_t = w_t - p_t$), domestic inflation, and employment (hours worked) for each individual country. The last two variables represent country-specific interest rates, which, although not directly pertinent to the main focus (being included purely for technical reasons), display a co-movement in line with the model's imposition of the monetary union. Generally, dynamic responses are very much in line with textbook literature \textcolor{red}{\parencite{jordigal_2015_monetary}}: An increase in interest rate makes saving more attractive, making households incentivised to consume less ``today'' in hopes to consume more ``tomorrow'' (see Euler equation). A decline in demand for home goods results in decline to output, which creates an abundance of labour, leading to an increase in unemployment. An increase in interest rate also leads to downward inflationary pressure, because goods become relatively less attractive to households compared to saving and making goods cheaper becomes the best strategy for price-setting, profit-maximising firms. Finally, given that employment decreases, bringing down the tax revenue, the government responds with an increase in labour tax rate ($\tau^{UK}_t$) to clear the government budget. All effects fade as economy returns to its steady state equilibrium.
\import{./Graphs/IRFs}{m_four.tex}

Next, dynamic responses to an increase in government spending are considered. Across all policy scenarios, the responses exhibit a consistent behaviour. That is, the government reduces the quantity of available resources to the household, leading to a decrease in consumption. The negative wealth effect becomes greater than the substitution effect between consumption and leisure, leading to an increase in employment. An increase in labour supply allows firms to produce more (output increases) while paying less in real terms (real wage decreases). In all scenarios, the supply of government issued bonds increases as the government raises funds to cover its spending. In the scenarios where labour tax exists, the government raises the tax rate, as well as issues bonds. The model considered in this dissertation assumes that the government ``consumes'' domestic goods. Therefore, an increase in government spending raises demand for these goods and creates an upward inflationary pressure, as shown in the Figure \textbf{?} below. Domestic goods become less competitive internationally, leading to depreciation in terms of trade (displayed only in Figure \textbf{?} in the interest of space). 
\import{./Graphs/IRFs}{f_one.tex}
\import{./Graphs/IRFs}{f_one_f.tex}
\import{./Graphs/IRFs}{f_two.tex}
\import{./Graphs/IRFs}{f_three.tex}
\import{./Graphs/IRFs}{f_three_f.tex}
\import{./Graphs/IRFs}{f_four.tex}
More specifically, under the policy scenario four ($G:1,\ \tau:1$), IRFs suggest that a 0.25\% deviation from the steady state government spending in any given quarter, results in 0.06\% in Scotland and 0.02\% in the rest of the UK deviation in output in the same quarter.\footnote{Note that the dissertation uses 25 basis points quarterly increase, so that the annualised increase would be 1 percentage point. This makes the results more comparable to those found in the literature.} Consumption decreases by 0.2\% in both countries, while employment increases by 0.01\% in Scotland compared to only 0.0025\% in the rest of the UK. Domestic inflation increased by 0.01\% and 0.0065\% in Scotland and the rest of the UK, respectively. Tables \ref{table:responses_one_three} and \ref{table:responses_two_four} contrast 200-period cumulative responses between countries for each of the policy scenarios (column ``Diff.'').\footnote{Note that all variables return to their steady-state levels within the first 50 periods. The choice of using 200 as an arbitrarily large number ensures that the full effect is accurately calculated even if extreme persistence parameters were chosen.} It also displays the changes in responses when the labour tax is introduced. In this model, the introduction of labour taxes is found to \textit{soften} the effect of expansionary fiscal policy for output, employment, and real wages (the variables still deviate from the steady state but less so).  The response in consumption remains unchanged, while inflation and interest rate responses are slightly amplified when the labour tax is introduced. Importantly, these findings hold true for both pairs of scenarios (1 \& 3 and 2 \& 4), with similar magnitudes observed.

Moreover, Tables \ref{table:responses_one_two} and \ref{table:responses_three_four} show how the cumulative effect changes when either country joins the fiscal union. For the rest of the UK, when there are no labour taxes, being in or out a fiscal union makes no difference. However, in policy scenarios 3 and 4, even though Scotland is small compared to the rest of the UK, it has a significantly higher steady state labour tax rate, which slightly changes the government spending dynamics for the fiscal union. Scotland is found to be more susceptible to such changes (with or without the existence of labour tax). More specifically, when Scotland joins the fiscal union, the responses are ``pulled'' towards those of the rest of the UK. Note, that for all variables in Tables \ref{table:responses_one_three}-\ref{table:responses_three_four}, the introduction of a labour tax has a significantly greater impact on dynamic responses compared to joining a fiscal union.
\import{./Results/Tables}{results_one_three.tex}
\import{./Results/Tables}{results_two_four.tex}
\newpage
\import{./Results/Tables}{results_one_two.tex}
\import{./Results/Tables}{results_three_four.tex}
In terms of transmission of shock from one country to the other, the model does not consider migration, national wages, and significant price pass-through (low real exchange rate between the two economies). Therefore, in policy scenarios with two governments, endogenous variables in one country do not respond considereably to shocks in government spending in another country; the effect exhibits a magnitutude on the order of 1e-13. This is a clear limitation of the model and it calls for additional examination.

Finally, Figures \ref{figure:sensitivity_analysis_phi_b}-\ref{figure:sensitivity_analysis_rho_g} illustrate the sensitivity of dynamic responses given changes in fiscal feedback and persistence parameters $\phi_b, \phi_g, \rho_g$ (and their rUK/UK counterparts). For each figure, one parameter values are allowed to vary while the other two are fixed to prior mean values. The sensitivity analysis is carried out for policy scenario four only. Unsurprisingly, the fiscal policy persistence parameter $\rho_g$ is found to have the greatest impact on the magnitude of responses, with some of the variables even responding in the opposite than expected direction when small $\rho_g$ values are considered. According to \textcite[27]{jordigal_2005_understanding}, the persistence parameter is associated with the strength of the wealth effect (the more persistent is the government spending shock, the greater wealth effect is imposed on the households). This is confirmed by Figure \ref{figure:sensitivity_analysis_rho_g}: as $\rho_g$ decreases, the dynamic responses return to their steady state values quicker. The positive responses of real wage when $\rho_g \in \{ 0.10, 0.60 \}$ require further attention but it could be attributed to the price stickiness and flexible wages: output and employment initial response is dramatically greater to their counterpart responses when $\rho_g = 0.99$, while initial response of inflation (period-to-period price level change) is almost unchanged for all $\rho_g$ values. This suggests that nominal wages increase more than sticky prices, resulting in positive real wage response. In line with \citereset\textcite[52]{jordigal_2005_understanding}, it can be shown that the cumulative effect on output is decreasing in $\phi_g$ but increasing in $\phi_b$. However, only the amount of bonds issued is \textit{considerably} sensitive to changes in $\phi_b$ and $\phi_g$.
\newpage
\import{./Graphs}{sensitivity_analysis_phi_b.tex}
\newpage
\import{./Graphs}{sensitivity_analysis_phi_g.tex}
\newpage
\import{./Graphs}{sensitivity_analysis_rho_g.tex}

    