\textcite{ricci_2019_essays} was the first to build a large-scale two-country DSGE model explicitly tailored to Scotland and the rest of the UK. In an attempt to retain the model's simplicity while still allowing policy analysis, this dissertation will primarily build on the work of \textcite{gali_2005_monetary} and \textcite{jordigal_2015_monetary}. In contrast, \textcite{ricci_2019_essays} model was based on the work of \textcite{rabanal_2010_eurodollar}, who were among the first to build a medium-to-large two-country DSGE model. Neither \textcite{gali_2005_monetary} nor \textcite{jordigal_2015_monetary} models considered lump-sum or distortionary taxes, or government spending, more generally. While extensive literature covers government spending in DSGE models, few to none cover government spending in a small open economy (SOE) NK DSGE model, and even fewer apply it to a two-country setting. Therefore, most of the derivations had to be carried out using a pen and paper, and step-by-step derivations are provided in the Appendix. Many of \citereset\textcite{jordigal_2015_monetary} derivations relied on the assumption that steaty stade output is equal to the state state consumption, i.e. $Y=C$. When the government term is introduced, then many of the expressions lose their inherent elegance and simplicity. The dissertation also considered a zero government consumption in the steady steady state assumption to recover algebraic simplicity;\enlargethispage{\baselineskip}\footnote{Log aggregate resource constraint $y_t = \left( 1 - \frac{G}{Y}\right)c_t - \frac{G}{Y}g_t$ can be rewritten to $y_t = c_t - \hat{g}_t$ using $\tilde{g}_t = G_t/Y$ instead of $g_t = (G_t - G)/G$.} however, historically (over the sample period of twenty years), Scotland had significantly higher government spending-to-output ratio, compared to that of the rest of the UK (see Application section). Thus, the dissertation opted for not using a simplifying assumption because a major source of representative asymmetry in dynamic responses would have been forgone.

However, the focus of this dissertation is not to build the most factually accurate model of Scotland or the United Kingdom but to assess the asymmetric responses in government spending when Scotland is part of the fiscal union and when it is fiscally autonomous. The fiscal union scenario refers to the Westminister government collecting taxes from all four countries of the UK and distributing them according to the Barnett formula. The fiscal autonomy scenario refers to the Holyrood government's ability to collect tax revenue, issue bonds (borrow), and spend it at its sole discretion. We further break down the scenarios by allowing public expenses to be funded by lump-sum and distortionary (labour) taxes. This brings the number of policy scenarios considered by the dissertation to four. In all four policy scenarios, the dissertation assumes that a single monetary authority sets one UK-wide interest rate.\footnote{Due to technical limitations, the dissertation modelled two countries as having individual nominal interest rates. However, in all applicable equations, a population weighted sum of the two interests rates was used. Conceptually, it is equivalent to having one UK-wide interest rate, where Scotland's interest rate ``influences'' UK-wide interest rate ($\varpi \approx 8.5\%$) but is primarily determined by the rest of the UK ($1-\varpi \approx 91.5\%$), so Scotland (almost) takes it as given.}

Finally, in line with most of the literature, variables referring to the home country (Scotland) will be denoted without an asterisk, i.e., $Y_t$, while foreign country (the rest of the UK) will be denoted with an asterisk, i.e., $Y^*_t$. $Y_t^{UK}$ and $Y_t^{W}$ will denote UK- and world-wide variables, respectively. Given that Scotland and the rest of the UK are modelled as symmetrical,\footnote{Equations can differ in parameter values but not structurally.} Sections 2.1-2.4 describe the model only for Scotland, but for each presented equation in the Sections, there exists a corresponding equation for the rest of the UK economy. In cases when this is not true, it will be stated explicitly.

\newpage
\import{./Theoretical DSGE model}{households.tex}
\newpage
\import{./Theoretical DSGE model}{government.tex}
\newpage
\import{./Theoretical DSGE model}{firms.tex}
\newpage
\import{./Theoretical DSGE model}{equilibrium.tex}