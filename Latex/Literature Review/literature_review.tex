\section{Literature Review}

On the 10th of May, 2023, the Monetary Policy Committee at the Bank of England gathered to discuss the latest international and domestic data on economic activity. Even though the Committee has a 2\% CPI target, the UK's economy had undergone a sequence of very large and unexpected shocks and disturbances, resulting in twelve-month CPI inflation above 10\%. The majority of the Committee members (78\%) believed that an increase in interest rate ``was warranted" \parencite[4]{boe_2023_monetary}, while the remaining members believed that the CPI inflation will ``fall sharply in 2023" \parencite[5]{boe_2023_monetary} as a result of the economy naturally adjusting to the effects of the energy price shocks. They feared that the preceding increases in the interest rate have not yet been internalised and raising the interest rate any further could result in a reduction of inflation ``well below the target" \parencite[5]{boe_2023_monetary}. This is an excellent illustration of the ``informal dimension of the monetary policy process", that \parencite[26]{gals_2007_macroeconomic} referred to in their work explaining modern macroeconomic models and new frameworks. According to them, while the informal dimension cannot be removed, we can build formal and rigorous models that would help the Committee and institutions-alike to understand ``objectives of the monetary policy and how the latter should be conducted in order to attain those objectives" \parencite[2]{jordigal_2015_monetary}. This task is not straightforward and has been central (albeit - fruitful) to most macroeconomic research in the past decades. The following section of the literature review will present a brief evolution of the study of business cycles. It will be followed by an overview of large macroeconomic models adopted by central banks and international organisations to illustrate the relevance of this research. The final part of this section will review the modelling of fiscal policy in DSGE models.

\textcite{snowdon_1994_a} guide contrasts differences between competing schools of thought of the the past century. It starts by discussing the intellectual debate between John Maynard Keynes (1883-1946) and the ``old'' (as opposed to New) classical economists. According to \citereset\textcite[42]{snowdon_1994_a} the ``Keynes v. Classics'' debate started in the 1930s and ``has continued in various forms ever since'' \parencite[42]{snowdon_1994_a}. \textcite{blanchard_2000_what} regards that period as the epoch of ``exploration'', as ``all the right ingredients, and quite a few more, were developed'' \parencite[1376]{blanchard_2000_what}. Indeed, the ideas of the time underpin mainstream schools of thought prevalent today: New Keynesianism and Real Business Cycle Theory \parencites[1]{jordigal_2015_monetary}[42]{snowdon_1994_a}. Both schools of thought agree that any capitalist market economy does not always produce at its equilibrium level. They do disagree about the origin and persistence of such deviations \parencite[43]{snowdon_1994_a}, as well as means for correcting them. For instance, the classical economists believed such deviations are possible only in the short-run because the market is efficient at restoring full employment equilibrium. The efficiency assumption stemmed from the belief that households and firms are rational and utility/profit maximisers, operating in perfectly competitive environment with flexible prices and complete knowledge of market conditions. The classical economists did believe that persistent and long-run deviations are possible but only due to inefficient and undesirable government interferance or monopolistic competition \parencite[43]{snowdon_1994_a}. In start contrast, Keynes questioned the ``self-equilibrating properties of the economy'' \parencite[89]{snowdon_1994_a} and the role the government. The central belief of Keynesian school of thought is that the aggregate level of output is determined by the aggregate demand, which depends on unstable and inevitably uncertain expectations of investment profitability \parencite[65]{snowdon_1994_a}. That is, in the \textit{General Theory}, Keynes (1936) attributed output fluctuations to erratic shocks in investment. Keynes (1936) also ``rejected the classical notion that interest was the reward for postponed current consumption'' \parencite[66]{snowdon_1994_a} and argued that the money liquidity preference\footnote{The demand for money holdings. A concept introduced by Keynes \parencite[66]{snowdon_1994_a}.} has much greater influence over the interest rate. When the assumption of nominal rigidities in prices (such as the costs associated with updating prices, often referred to as price stickiness) is considered, monetary policy emerges as a viable tool that, according to Keynesians, can be utilized to stimulate aggregate demand. Keynesians believe that the government and central bank should counter-cyclically employ fiscal and monetary policies, respectively, to ensure more rapid return to full employment and stability.\footnote{Counter-cyclical fiscal policy means increasing government spending, relaxing regulations, reducing taxation, etc. when the aggregate output is decreasing (and vice-versa). Similarly, counter-cyclical monetary policy means ``raising'' nominal interest rate when aggregate output is increasing and vice-versa.}

\textcite{blanchard_2000_what} offers a compact description of macroeconomic research in the twentieth century. In their panegyric and optimistic essay, the researcher argues that the century can be divided into three epochs based on the prevailing beliefs about the economy and frameworks of the time: Pre-1940, From 1940 to 1980, and Post-1980. 
\import{./Graphs}{timeline.tex}
In the 1980s, \textcite{kydland_1982_time} and \textcite{prescott_1986_theory} published seminal papers on the Real Business Cycles (RBC) theory. According to \textcite[2]{jordigal_2015_monetary}, frameworks presented in the papers ``provided the main reference'' and firmly established the use of dynamic stochastic general equilibrium (DSGE) models as crucial tools for macroeconomic analysis. The models allow quantitative analysis and incorporation of data either via calibration or estimation of parameters. \textcolor{red}{TBC}

International Monetary Fund (IMF)

\textbf{Government in DSGE literature review}

To begin with, RBC models predict a negative response in consumption following an increase in government spending. More specifically, government spending is modelled to absorb resources, which makes households worse off and incentivises more hours worked. Greater labour supply for any given wage reduces firms' marginal cost and induces output \parencite[319]{baxter_1993_fiscal}. That is, consumption, conditional on shocks in government spending, is countercyclical. Keynesian models, in stark contrast, predict the opposite. The countercyclicality is why the DSGE models sometimes do not consider government spending. 

Empirically, the findings of the Keynesian models are more in line with the observed macroeconomic patterns. For instance, \textcite{blanchard_2002_an} performed a VAR analysis on the dynamics of consumption and government spending. They built six structural VAR models, one for each component of GDP: output, consumption, government spending, investment, export, and import. The two other variables were taxes and government spending/output\footnote{That is, if the GDP component of interest is government spending, then the second variable is output; in all other cases, the second variable is government spending.}. The key finding of the analysis is that government spending has a positive effect on consumption. 

\textcite{jordigal_2005_understanding} show that NK DSGE models can be ``recovered'' by assuming that households have limited access to financial markets/saving technologies or are poor (they consume all of their labour income). Households that smooth their consumption by saving are often regarded as Ricardian households, while those that do not - non-Ricardian or \textit{hand-to-mouth} households. Some of the latest literature NK DSGE literature models both types of households explicitly\footnote{In fact, there exists literature with more than two types of households. For instance, a recent paper by \textcite{eskelinen_2021_monetary} models poor hand-to-mouth, wealthy hand-to-mouth, and non-hand-to-mouth households.} with their ratio determined by a time-invarying exogenous coefficient. Arguably, such modelling would allow an improved fit of data. However, the key focus of this dissertation is the four policy scenarios, all of which heavily depend on governments' abilities to issue bonds and borrow. Hand-to-mouth households do not borrow/save, rendering bonds purposeless. While modelling hand-to-mouth households is even easier than the Ricardian households, modelling both types of households would drastically increase the complexity of the model, given the two-country and four policy scenarios setting. The absence of hand-to-mouth households is discussed in the limitations section.