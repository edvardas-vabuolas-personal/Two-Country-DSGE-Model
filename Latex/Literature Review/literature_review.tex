\section{Literature Review}

On the 10th of May, 2023, the Monetary Policy Committee at the Bank of England gathered to discuss the latest international and domestic data on economic activity. Even though the Committee has a 2\% Consumer Price Index (CPI) inflation target, the UK's economy had undergone a sequence of very large and unexpected shocks and disturbances, resulting in twelve-month CPI inflation above 10\%. The majority of the Committee members (78\%) believed that an increase in interest rate ``was warranted" \parencite[4]{boe_2023_monetary}, while the remaining members believed that the CPI inflation would ``fall sharply in 2023" \parencite[5]{boe_2023_monetary} as a result of the economy naturally adjusting to the effects of the energy price shocks. They feared that the preceding increases in the interest rate have not yet been internalised and raising the interest rate any further could result in a reduction of inflation ``well below the target" \parencite[5]{boe_2023_monetary}. This is an excellent illustration of the ``informal dimension of the monetary policy process", that \parencite[26]{gals_2007_macroeconomic} referred to in their work explaining modern macroeconomic models and new frameworks. According to them, while the informal dimension cannot be removed, we can build formal and rigorous models that would help the Committee and institutions alike understand fiscal and monetary policy objectives and how these should be conducted. This task is not straightforward and has been central (albeit - fruitful) to most macroeconomic research in the past decades. 

\subsection{From Past Debates to Modern Frameworks}
\textcite{snowdon_1994_a} guide contrasts differences between competing schools of thought of the past century. It discusses the intellectual debate between John Maynard Keynes (1883-1946) and the ``old'' (as opposed to New) classical economists. According to \citereset\textcite[42]{snowdon_1994_a} the ``Keynes v. Classics'' debate started in the 1930s and ``has continued in various forms ever since'' \parencite[42]{snowdon_1994_a}. \textcite{blanchard_2000_what} regards that period as the epoch of ``exploration'', as ``all the right ingredients, and quite a few more were developed'' \parencite[1376]{blanchard_2000_what}. Indeed, the ideas of the time underpin mainstream schools of thought prevalent today: New Keynesianism and Real Business Cycle Theory \parencites[1]{jordigal_2015_monetary}[42]{snowdon_1994_a}. Both schools of thought agree that any capitalist market economy does not always produce at its equilibrium level. They do disagree about the origin and persistence of such deviations \parencite[43]{snowdon_1994_a}, as well as means for correcting them. For instance, classical economists believed such deviations are possible only in the short run because the market is efficient at restoring full employment equilibrium. The efficiency assumption stemmed from the belief that households and firms are rational and utility/profit maximisers, operating in a perfectly competitive environment with flexible prices and complete knowledge of market conditions. The classical economists did believe that persistent and long-run deviations are possible but only due to inefficient and undesirable government interference or monopolistic competition \parencite[43]{snowdon_1994_a}. In stark contrast, Keynes questioned the ``self-equilibrating properties of the economy'' \parencite[89]{snowdon_1994_a} and the role of the government. The central belief of the Keynesian school of thought is that the aggregate level of output is determined by the aggregate demand, which depends on unstable and inevitably uncertain expectations of investment profitability \parencite[65]{snowdon_1994_a}. That is, in the \textit{General Theory}, \textcite{keynes_1936_general}, cited in \citereset\textcite[65]{snowdon_1994_a}, attributed output fluctuations to erratic shocks in investment. Keynes also ``rejected the classical notion that interest was the reward for postponed current consumption'' \parencite[66]{snowdon_1994_a} and argued that the money liquidity preference\footnote{The demand for money holdings. A concept introduced by Keynes \parencite[66]{snowdon_1994_a}.} has much greater influence over the interest rate.\footnote{According to \textcite[1380]{blanchard_2000_what}, general equilibrium models suggest that both ``loanable funds'' and ``liquidity preference'' determine the interest rate, i.e. the ``truth'' is somewhere between the classical and Keynesian schools of thought.} When the assumption of nominal rigidities in prices (such as the costs associated with updating prices, often referred to as price stickiness) is considered, monetary policy emerges as a viable tool that, according to Keynesians, can be utilised to stimulate aggregate demand. Keynesians believe that the government and central bank should counter-cyclically employ fiscal and monetary policies, respectively, to ensure a more rapid return to full employment and stability.\footnote{Counter-cyclical fiscal policy means increasing government spending, relaxing regulations, reducing taxation, etc. when the aggregate output decreases (and vice-versa). Similarly, counter-cyclical monetary policy means ``raising'' nominal interest rate when aggregate output increases and vice-versa.}

Intellectual debates and ``exploration'' was followed by an epoch of ``consolidation'' \textcite{blanchard_2000_what}. In their panegyric and optimistic essay, the \citereset\textcite{blanchard_2000_what} argues that the period between 1940 and 1980 was the ``golden age of macroeconomics'' \parencite[1379]{blanchard_2000_what} as the progress was ``fast and visible''. \textcite{hicks_1937_mr} famously formalised ideas of the Keynesian economists and developed a widely known IS-LM model\footnote{The abbreviation stands for ``investment-saving'' and ``liquidity preference-money supply''} that ``integrates real and monetary factors in determining aggregate demand and therefore the level of output and employment'' \parencite[90]{snowdon_1994_a}. Over a few decades, the basic model or ``skeleton apparatus'' \parencite[158]{hicks_1937_mr} was developed in many directions. By the mid-1970s, it was able to accommodate a multitude of extensions, including ``backward'' and ``forward'' looking variables, rational expectations, and the Phillips curve linking inflation to unemployment \parencite[1382]{blanchard_2000_what}. However, according to \citereset\textcite[1382]{blanchard_2000_what}, the treatment of imperfections (deviations from the economy envisaged by the ``old'' classical school of thought economists) was too ``casual'' (superficial) and in the mid-1970s, \textcite{sargent_1973_rational} prompted an ``intellectual crisis'' \parencite[1382]{blanchard_2000_what} by showing that rational expectations of inflation, combined with Keynesian models, suggest the minimal effect of money on output. Following this, one camp of economists began exploring even ``deeper'' market imperfections and are aptly referred to as ``New Keynesians''. In contrast, the other camp explored models without any imperfections and developed models explaining business cycles primarily as fluctuations in the technology level. The former economists are referred to as ``Real Business Cycle'' (RBC) theorists. Since \textcites{kydland_1982_time}{prescott_1986_theory}, the role of monetary policy within the RBC framework became the primary reference and ``to a large extent the core of macroeconomic theory'' \parencite[2]{jordigal_2015_monetary}. The RBC theorists' developed dynamic stochastic general equilibrium (DSGE) models proved essential and are widely adopted by researchers and economists.

Finally, ideas from both camps converged in less than twenty years, according to \textcite[1388]{blanchard_2000_what}, and most modern macroeconomic research focuses on imperfections in goods, labour, and financial markets using New Keynesian models \parencite{jordigal_2015_monetary}. To summarise, according to \textcite{blanchard_2000_what}, the macroeconomic research of the past century can be divided into three epochs based on the prevailing beliefs about the economy and frameworks of the time: Pre-1940, From 1940 to 1980, and Post-1980.
\import{./Graphs}{timeline.tex}

\subsection{Application of DSGE Models}
As mentioned, DSGE models became the standard among researchers and macroeconomists in the public and private sectors. \textcite{smets_2003_an} built a large-scale DSGE model tailored to the Euro area economy, and \textcite[595]{smets_2007_shocks} findings show that the theory-driven DSGE model exhibit forecasting performance akin to that of an a-theoretic (econometric) Bayesian VAR (with Minnesota-type prior) model. Following \textcite{adolfson_2007_bayesian}, many recent models take into account the world economy and allow international trading. Two-country models were derived to allow examination of cross-country business cycles or the effect of business cycles originating in a foreign economy; examples include \textcite{kolasa_2009_structural}, who used a two-country DSGE model to analyse the potential effects of Poland joining the Euro Area. \textcite{gregorydewalque_2017_an} and \textcite{gunter_2017_estimating} analysed exchange rate dynamics using a Euro Area-US model. International Monetary Fund (IMF) uses ``GEM'', a large-scale multi-country DSGE model for \textit{World Economic Outlook} analyses \parencite{mrivantchakarov_2004_gem}; \textcite{herv_2011_the} describes OECD's global model, while \textcite{albonico_2019_the} presents \textit{The Global Multi-Country Model} (GM), an estimated DSGE model for each Euro Area country, used by the European Central Bank. \citereset\textcite[3]{albonico_2019_the} lists DSGE models developed by national central banks tailored to their respective countries. At the same time, a recent survey by \textcite{yagihashi_2020_dsge} provides a quantitative and qualitative comparison of 84 DSGE models developed by policy institutions around the world.

\subsection{Fiscal Policy in DSGE Models}

To begin with, RBC models predict a negative response in consumption following an increase in government spending. More specifically, government spending is modelled to absorb resources, which makes households worse off and incentivises more hours worked. Greater labour supply for any given wage reduces firms' marginal cost and induces output \parencite[319]{baxter_1993_fiscal}. That is, consumption, conditional on shocks in government spending, is countercyclical. Keynesian models, in stark contrast, predict the opposite.

Empirically, the findings of the Keynesian models are more in line with the observed macroeconomic patterns. For instance, \textcite{blanchard_2002_an} performed a VAR analysis on the dynamics of consumption and government spending. They built six structural VAR models, one for each component of GDP: output, consumption, government spending, investment, export, and import. The key finding of the analysis is that government spending has a positive effect on consumption. 

\textcite{jordigal_2005_understanding} show that NK DSGE models can be ``recovered'' by assuming that households have limited access to financial markets/saving technologies or are poor (they consume all of their labour income). Households that smooth their consumption by saving are often regarded as Ricardian households, while those that do not are referred to as non-Ricardian or \textit{hand-to-mouth} households. Some of the latest NK DSGE literature models both types of households explicitly\footnote{In fact, there exists literature with more than two types of households. For instance, a recent paper by \textcite{eskelinen_2021_monetary} models poor hand-to-mouth, wealthy hand-to-mouth, and non-hand-to-mouth households.} with their ratio determined by a time-invarying exogenous coefficient. Arguably, such modelling would allow an improved fit of data. While modelling hand-to-mouth households is algebraically simpler than modelling the Ricardian households, modelling both types of households would drastically increase the complexity of the model, given the two-country and four policy scenarios setting. The absence of hand-to-mouth households is discussed in the limitations section.