There is more than one method to estimate DSGE model parameters, and their theoretical comparisons are provided in \textcite[123-130]{an_2007_bayesian}, as well as \textcite[109-110]{ricci_2019_essays}. Ever since \textcite{schorfheide_2000_loss} applied Random-Walk Metropolis (RMW) algorithm to estimate DSGE models, the method gained popularity and is the most prevalent estimation method in recent literature. 

The algorithm aims to approximate the posterior density function using a large number of parameter draws (sets of values). The domain of the candidate values is pre-specified using prior distributions and combined with a vector of time series to produce \textit{likelihood}. Likelihood values of two sequential draws are used to calculate the acceptance probability for the candidate value. A large number of accepted draws form the posterior density function. Due to the limitations of the scope of this dissertation, the specifics of the RMW algorithm will not be delved into, but an interested reader should see \textcite[131]{an_2007_bayesian} for a formal treatment of the algorithm and \textcite{blake_2017_applied} handbook of applied Bayesian econometrics that has a dedicated section for implementing the algorithm for DSGE models and even includes a complementary Matlab code repository. This dissertation will utilise out-of-the-box solutions offered by the Dynare computer package \parencite{adjemian_2021_dynare} to obtain parameter draws.

While estimating all parameters is an option, in theory, most parameter values will be calibrated in line with \textcite[41]{harrison_2010_evaluating}, who built a DSGE model for the United Kingdom and provides detailed reasoning behind calibrated values, and \textcites[67-75]{jordigal_2015_monetary}[3839]{gali_2005_monetary} for less important parameters. The reason behind the decision to calibrate the majority of the parameters is the simplicity of our model. The model does not consider capital, habit formation, sticky wages, limited access to financial instruments, and many other extensions in the literature that improve data fit \parencite{yagihashi_2020_dsge}. This makes our data uninformative for some parameters, and estimation results heavily rely on priors. Table 8 below lists calibrated (as opposed to estimated) model parameters for Scotland and the rest of the UK. All but fiscal policy parameters and steady-state ratios are assumed to be symmetrical. 
\import{./Application/Tables}{model_parameters.tex}

The dissertation, however, does estimate three sensitive parameters key to the research question: $\phi_b$, $\phi_g$, and $\rho_g$ (and their rUK / UK counterparts). The acceptance rate per chain was c. 22\%, close to the optimal acceptance rate suggested by \textcite{roberts_2001_optimal}. The acceptance rate in the range of 23\% ensures that the variance of candidate parameter values is neither too large nor too small \parencite{roberts_2001_optimal}, meaning that all sets of parameters had a reasonable probability of being drawn. The number of MH draws and initial (``burn-in'') draws was set to 300,000 and 100,000, respectively. Figures (\ref{fig:estimation_results_scot}, \ref{fig:estimation_results_ruk}, and \ref{fig:estimation_results_uk}) display how the 300,000 draws varied. The variance of the draws suggests that the serial correlation of lagged draws was not persistent (fading), which is also indicative of successful estimation \parencite{roberts_2001_optimal}. The posterior means of all estimated parameters are close to their prior means, which can be primarily attributed to tight prior variance. However, the two means do not coincide, suggesting (to some degree) an informative estimation of the parameters. In fact, estimation of $\phi_b$ and $\phi_g$ are bound to be difficult given that the fiscal rule, while being reasonable and intuitive, is imposed arbitrarily to ``close'' the model. The true rule followed by the decision-makers, however, might be more complex and considerate of other metrics not captured by the model. All parameter values were drawn from the Beta distribution due to its values being bounded by zero and one. Table 9 lists shape parameters, implied prior means and standard deviations, as well as posterior means of all estimated parameters. 
\import{./Application/Tables}{priors_posteriors.tex}
\import{./Graphs}{MCMC_SCOT.tex}
Figures (\ref{fig:estimation_results_ruk} and \ref{fig:estimation_results_uk}) displaying draws of corresponding variables for the rest of the UK and UK can be found in Appendix. In terms of data, most time series for Scotland and the rest of the UK were acquired from the Quarterly National Accounts of Scotland (QNAS) 2022 and the Quarterly Nationals Account (QNA) 2023, respectively. In contrast to \textcite{ricci_2019_essays}, who used the 1998Q1-2007Q4 dataset, this dissertation utilised the full QNA/QNAS dataset (1998Q1-2021Q4) for the variables of interest.\footnote{\textcite{ricci_2019_essays} argued that their DSGE model is not ``equipped to account for''\parencite[123]{ricci_2019_essays} the global financial crisis of 2008. However, using a full dataset was found to improve estimation results for this model and is a more transparent choice. Therefore, the dissertation opted for the 1998Q1-2021Q4 sample date range.} Each time series was processed in line with \textcite{pfeifer_2018_a}. That is, the time series for Scotland were adjusted to 2023 price levels and divided by the number of (estimated) working-age residents in Scotland. Then, the time series were expressed in natural logarithms and first-differenced to induce stationarity. Finally, the series was demeaned in line with the model specification. Identical data processing steps were taken for the rest of the UK time series. Note that data was not available for all regions of the UK. Therefore, in most cases, the rest of the UK time series were calculated by taking Scotland's time series and subtracting them from the UK-wide ones. The resulting values should be approximately equal to the rest of the UK ones, as long as QNAS and QNA use symmetrical accounting/data processing methods for each pair of time series. Figures \ref{fig:original_series} and \ref{fig:processed_series} display the pre- and post-transformed time series line charts, respectively.
\import{./Application/Tables}{sources.tex}