This section will consider the responses of endogenous variables, given two types of shocks: monetary and fiscal. Although the examination of monetary shocks falls beyond the scope of this dissertation, it serves as a valuable benchmark to assess the accuracy and validity of our model. The Appendix Z also includes dynamic responses to world output and price level, as well as preferences ($z_t$). This dissertation does not provide a discussion on responses to these shocks. For the monetary policy shock, we use the policy scenario four model, i.e. a single fiscal government that uses labour tax and bond-issuance as fiscal instruments to raise budget revenue. In the interest of space, majority of the impulse reponse function (IRF) plots are placed in the Appendix Z. 

Figure \textbf{?} displays responses of fourteen endogenous variables. Two of the variables are UK-wide, namely labour tax rate and interest rate itself. Ten of the parameters are consumption, output, real wage ($wp_t = w_t - p_t$), domestic inflation and employment (hours worked) for each country. The remaining two parameters are country-specific interest rates. Whilst not of interest directly (they exist purely for technical reasons), it shows that the two interest rates co-move in line with our monetary union imposition. Other responses are very much in line with textbook literature (\textbf{refs}). An increase in interest rate makes saving more attractive, making households incentivised to consume less ``today'' in hopes to consume more ``tomorrow'' (see Euler equation). A decline in demand for home goods results in decline to output, which creates an abundance of labour, leading to an increase in unemployment. An increase in interest rate also leads to downward inflationary pressure, because goods become relatively less attractive to households compared to saving and making goods cheaper becomes the best strategy for price-setting, profit-maximising firms. Finally, given that employment decreases, bringing down the tax revenue, the government responds with an increase in labour tax rate ($\tau^{UK}_t$) to clear the government budget. All effects fade as economy returns to its steady state equilibrium.
\import{./Graphs/IRFs}{monetary_four.tex}
Note: plots that have a single line but two elements ('rUK' and 'Scotland') inside a legend mean that responses are identical (the two lines are overlapping).


% \newpage
% \import{./Graphs/IRFs}{fiscal_one.tex}
% \newpage
% \import{./Graphs/IRFs}{fiscal_one_f.tex}
% \newpage
% \import{./Graphs/IRFs}{fiscal_two.tex}
% \newpage
% \import{./Graphs/IRFs}{fiscal_three.tex}
% \newpage
% \import{./Graphs/IRFs}{fiscal_three_f.tex}
% \newpage
% \import{./Graphs/IRFs}{fiscal_four.tex}