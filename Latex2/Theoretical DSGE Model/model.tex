\textcite{ricci_2019_essays} was the first to build a large-scale two-country DSGE model explicitly tailored to Scotland and the rest of the UK. In an attempt to retain the model's simplicity while still allowing policy analysis, this dissertation will primarily build on the work of \textcite{gali_2005_monetary} and \textcite{jordigal_2015_monetary}. In contrast, \textcite{ricci_2019_essays} model was based on the work of \textcite{rabanal_2010_eurodollar}, who were among the first to build a medium-to-large two-country DSGE model. Neither \textcite{gali_2005_monetary} nor \textcite{jordigal_2015_monetary} models considered lump-sum or distortionary taxes, or government spending, more generally. While extensive literature covers government spending in DSGE models, few to none cover government spending in a small open economy (SOE) NK DSGE model, and even fewer apply it to a two-country setting. Therefore, most of the derivations had to be carried out using a pen and paper, and step-by-step derivations are provided in the Appendix. Many of \textcite{jordigal_2015_monetary} derivations relied on the assumption that steaty stade output is equal to the state state consumption, i.e. $Y=C$. When the government term is introduced, then many of the expressions lose their inherent elegance and simplicity.

Moreover, the focus of this dissertation is not to build the most factually accurate model of Scotland or the United Kingdom but to assess the asymmetric responses in government spending under fiscal autonomy and fiscal union. The fiscal union scenario refers to the Westminister government collecting taxes from all four countries of the UK and distributing them according to the Barnett formula. The fiscal autonomy scenario refers to the Holyrood government's ability to collect tax revenue, issue bonds (borrow), and spend it at its sole discretion. We further break down the scenarios by allowing public expenses to be funded by lump-sum and distortionary (labour) taxes. This brings the number of policy scenarios considered by the dissertation to four. In all four policy scenarios, the dissertation assumes that a single monetary authority sets one UK-wide interest rate.\footnote{The Dynare code complementing this dissertation allows the user to switch between the assumption of monetary independence and union, as well as choosing a policy scenario of interest. The results section does not consider impulse response functions under monetary independence, as it is outside the scope of the dissertation.}

Finally, in line with most of the literature, variables referring to the home country (Scotland) will be denoted without an asterisk, i.e., $Y_t$, while foreign country (the Rest of the UK or \textbf{rUK}) will be denoted with an asterisk, i.e., $Y^*_t$. Population-weighted sums of these variables will be referred to as UK-wide variables and denoted as $Y_t^{UK}$. Given that Scotland and the rest of the UK are modelled as symmetrical, Sections 2.1-2.4 describe the model only for Scotland, but note that for each presented equation in the Sections, there exists a corresponding equation for the rest of the UK economy. In cases when this is not true, it will be stated explicitly.