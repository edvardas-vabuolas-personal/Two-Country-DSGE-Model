\subsection{Equilibrium}
The goods market for a specific good $j$ clears when domestic firms produce just enough of the good to satisfy the demand of home households, foreign households, and the home government. In line with \textcite{jordigal_2015_monetary}, the demand for exports of good $j$ is taken to be given as:
\begin{align}
    X_t(j) &= \left(\frac{P_{H,t(i)}}{P_{H,t}}\right)^{-\epsilon} X_t\\
    \text{where} \quad X_t &= \left( \int_{0}^{1} X_t(j)^{\frac{\epsilon-1}{\epsilon}} \, dj\right)^{\frac{\epsilon}{\epsilon-1}} \label{eq:index_of_aggregate_exports}\\
    & =\upsilon\left( \frac{P_{H,t}}{\mathcal{E}_t \xbar{P}_{H,t}}\right)^{-\eta}Y^\star_t \label{eq:aggregate_exports}\\
    & = \upsilon \mathcal{S}_t^{\eta}Y^\star_t \label{eq:aggregate_exports_terms_of_trade}
\end{align} 
where (\ref{eq:index_of_aggregate_exports}) is the index of aggregate exports, (\ref{eq:aggregate_exports}) determines aggregate exports as a function of world output (the relationship is assumed to be given), and (\ref{eq:aggregate_exports_terms_of_trade}) is derived by substituting definition of the effective terms of trade. Differently than \textcite{jordigal_2015_monetary}, we introduce index of government purchasing:
\begin{equation}
    G_t = \left( \int_{0}^{1} {G_t(j)}^{\frac{\epsilon-1}{\epsilon}} \, dj\right)^{\frac{\epsilon}{\epsilon-1}}
\end{equation}
so that the government demand of any good $j$ is defined as (derivation provided by Appendix A.xx-A.xx):
\begin{equation}
    G_t(j) = \left( \frac{P_{H,t}(j)}{P_{H,t}} \right)^{-\epsilon} G_t
\end{equation}
Therefore, total demand for good $j$ is:
\begin{align}
    Y_t(j) &= C_t(j) + X_t(j) + G_t(j) \label{eq:demand_for_one_good}\\
    \vdots & \quad \text{(see Appendix A.xx-A.xx)} \\
    Y_t(j) &= \left(\frac{P_{H,t}(j)}{P_{H,t}}^{-\varepsilon}\right) \left[ (1-\upsilon)\left(\frac{P_{H,t}}{P_t}\right)^{-\eta}C_t + \upsilon \mathcal{S}^\eta Y^\star_t + G_t\right]
\end{align}
which can be plugged in to definition of aggregate output $Y_t = \left( \int_{0}^{1} Y_t(j)^{\frac{\varepsilon-1}{\varepsilon}}\right)^{\frac{\varepsilon}{\varepsilon-1}}$ to yield:
\begin{equation}
    Y_t = (1-\upsilon) \left(\frac{P_{H,t}}{P_t}\right)^{-\eta}C_t + \upsilon \mathcal{S}^\eta Y^\star_t + G_t
\end{equation}
Note that Equation (\ref{eq:demand_for_one_good}) and all subsequent derivations (marginally) vary depending on the policy scenario in question:
\begin{table}[H]
    \renewcommand{\arraystretch}{2}
    \centering
    \begin{tabular}{l|l|c}
    \makecell{Scen. 1 \& Scen. 3\\ G: 2, $\tau \in \{0, 1\}$} & \makecell{Scot. \\ rUK } & 
        \makecell{
            $Y_t(j) = C_t(j) + X_t(j) + G_t(j)$\\
            $Y^*_t(j) = C^*_t(j) + X^*_t(j) + G^*_t(j)$
        }  \\ 
        \makecell{Scen. 3 \& Scen. 4\\ G: 1, $\tau \in \{0, 1\}$} & \makecell{Scot. \\ rUK } & 
        \makecell{
            $Y_t(j) = C_t(j) + X_t(j) + \varpi G^{UK}_t(j)$\\
            $Y^*_t(j) = C^*_t(j) + X^*_t(j) + (1-\varpi) G^{UK}_t(j)$
        }   
    \end{tabular}
    \caption{Demand for good $j$ for different policy scenarios}
\end{table}
Equation (\ref{eq:demand_for_one_good}) can be loglinearised around a symmetric steady state to yield:
\begin{align}
    Y_t &= (1-\upsilon)\left(\frac{P_{H,t}}{P_t}\right)^{-\eta}C_t + \upsilon \mathcal{S}_t^{\eta}Y_t^* + G_t\\
    Y \mathbf{e}^{y_t} &= (1-\upsilon)\left(\frac{P}{P_{H}}\right)^{\eta}C \mathbf{e}^{-\eta p_{H,t} + \eta p_t + c_t}+ \upsilon S^{\eta} Y^* \mathbf{e}^{\eta s_t + y^*_t} + G \mathbf{e}^{g_t}\\
    \vdots & \quad \text{(see Appendix \ref{eq:appendix_log_rc_beginning} - \ref{eq:appendix_log_rc_end})} \nonumber \\
    y_t &= (1-\upsilon)c_t + \upsilon (2-\upsilon)\eta s_t + \upsilon y^*_t + G_Y g_t \label{eq:aggregate_rc}
\end{align}
where the last term is equal to zero if the government is financed via lump-sum taxes. (\textbf{??}) We can use Equation (\ref{eq:link_between_consumption_and_world_output}) that links domestic consumption to world output and previous equation (\ref{eq:aggregate_rc}) to express terms of trade as a function of domestic output, world output, preference shifter, and government spending:
\begin{align}
    y_t &= (1-\upsilon)\left( y^*_t + \frac{1}{\sigma}z_t + \frac{1-\upsilon}{\sigma}s_t \right) + \upsilon (2-\upsilon)\eta s_t + \upsilon y^*_t + G_Y g_t \\
    \vdots & \quad \text{(see Appendix \ref{eq:terms_of_trade_derivation_beginning} - \ref{eq:terms_of_trade_derivation_end})} \nonumber \\
    s_t &= \sigma_\upsilon(y_t - y^*_t) - (1-\upsilon)\Phi z_t - \sigma_\upsilon G_Y g_t \label{eq:terms_of_trade_w_gov}
\end{align}
where $\varpi = \sigma \eta + (1-\upsilon)(\sigma \eta - 1)$, $\Phi = \frac{1}{1 + \upsilon (\varpi - 1)}$ and $\sigma_\upsilon = \sigma \Phi$. Negative government term implies that (effective) trade of terms are decreasing in government spending. This is intuitive: the government is modelled to demand exclusively domestic goods, which induces inflationary pressure and makes home goods less competitive internationally. The opposite is true for the domestic-world output gap: if our firms produce relatively more than the rest of the world and is able to export relatively more, then terms of trade increase (the first term of (\ref{eq:terms_of_trade_w_gov}) is positive).

Furthermore, the Euler equation in (\ref{eq:euler_equation}) is a function of CPI, but using CPI definition (\ref{eq:cpi_inflation_tot}), it can be rewritten to be a function of domestic inflation and terms of trade:
\begin{align}
    c_t &= \E\{ c_{t+1}\} - \frac{1}{\sigma}(i_t - \E \{ \pi_{H,t+1}\} - \rho) + \frac{\upsilon}{\sigma}\E \{ \Delta s_{t+1} \} + \frac{1}{\sigma}(1-\rho_z)z_t \label{eq:euler_w_domestic_inflation}
\end{align}
Finally, aggregate resource constraint (\ref{eq:aggregate_rc}), terms of trade definition (\ref{eq:terms_of_trade_w_gov}), and new Euler equation (\ref{eq:euler_w_domestic_inflation}) can be combined and used to derive a version of dynamic IS equation:
\begin{align}
    c_t &= \E\{ c_{t+1}\} - \frac{1}{\sigma}(i_t - \E \{ \pi_{H,t+1}\} - \rho) + \frac{\upsilon}{\sigma}\E \{ \Delta s_{t+1} \} + \frac{1}{\sigma}(1-\rho_z)z_t \\
    \vdots & \quad \text{(see Appendix \ref{eq:dynamic_is_beginning}-\ref{eq:dynamic_is_end})} \nonumber \\
    y_t  &= \E_t \{y_{t+1}\} - \frac{1}{\sigma_\upsilon}(i_t - \E \{ \pi_{H,+1}\} - \rho) + \upsilon (\varpi - 1)\E \{\Delta y_{t+1}^* \} \nonumber \\ 
    &+ \frac{1-\upsilon}{\sigma} (1-\rho_z)z_t + \frac{\upsilon \varpi  - 1}{1-\upsilon} G_Y \{\Delta g_{t+1}\} \label{eq:a_version_of_dis}
\end{align}
which implies that output in current period depends not only on expected output, inflation, and change in world output, but it also depends on expected government spending. Equation (\ref{eq:a_version_of_dis}) can be expressed in terms of output and real interest rate gaps:
\begin{align}
    y^n_t  &= \E_t\{y^n_{t+1}\} - \frac{1}{\sigma_\upsilon}(r^n_t - \rho) + \upsilon (\varpi - 1)\E \{\Delta y_{t+1}^* \}  + \frac{1-\upsilon}{\sigma} (1-\rho_z)z_t \nonumber \\ 
    &+ \frac{\upsilon \varpi  - 1}{1-\upsilon} G_Y \E_t \{ \Delta g_{t+1} \}\nonumber \\
    \vdots & \quad \text{(see Appendix \ref{eq:final_dynamic_is_derivation_beginning} - \ref{eq:appendix_natural_rate_of_interest_dynamic_is})} \nonumber \\
    r^n_t  &= \sigma_\upsilon \E_t \{\Delta y^n_{t+1}\} + \rho + \sigma_\upsilon \upsilon (\varpi - 1)\E \{\Delta y_{t+1}^* \}  + \Phi (1-\upsilon)(1-\rho_z)z_t \nonumber \\ 
    &+ \sigma_\upsilon \frac{\upsilon \varpi  - 1}{1-\upsilon} G_Y \E_t \{ \Delta g_{t+1} \} \label{eq:natural_rate_of_interest_dynamic_is}\\
    \vdots & \quad \text{(see Appendix \ref{eq:appendix_natural_rate_of_interest_dynamic_is} - \ref{eq:final_dynamic_is_derivation_end})} \nonumber \\
    \tilde{y}_t &= \E_t \{\tilde{y}_{t+1}\} - \frac{1}{\sigma_\upsilon}(i_t - \E_t \{ \pi_{H,+1}\} - r^n_t)
\end{align}
where $\E_t\{y^n_{t+1}\}$ is the expected natural output and Equation (\ref{eq:natural_rate_of_interest_dynamic_is}) defines the natural real rate of interest $r_t^n$.
