``No. of governments'': Number of governments that can issue bonds and accrue debt to fund public services. In all scenarios, government spending is assumed to be an AR(1) exogenous process.
\begin{enumerate}
    \item Scenario 1
    \begin{enumerate}
        \item No. of governments: 1
        \item Public expenses funded by (wasteful) \underline{lump-sum tax}
    \end{enumerate}
    \item Scenario 2
    \begin{enumerate}
        \item No. of governments: 2
        \item Public expenses funded by (wasteful) \underline{lump-sum tax}
    \end{enumerate}
    \item Scenario 3
    \begin{enumerate}
        \item No. of governments: 1
        \item Public expenses funded by (distortionary) \underline{income tax}
    \end{enumerate}
    \item Scenario 4
    \begin{enumerate}
        \item No. of governments: 2
        \item Public expenses funded by (distortionary) \underline{income tax}
    \end{enumerate}
\end{enumerate}
\pagebreak
\begin{table}[H]
    \centering
    \begin{tabular}{c|c}
        \makecell{
            S HBC \\
            rUK HBC \\
            S GBC \\
            rUK GBC \\
            S TR \\
            rUK TR \\
            S RC \\
            rUK RC \\
        } & 
        \makecell{
            Household (Scotland) budget constraint \\
            Household (rUK) budget constraint \\
            Government (Holyrood) budget constraint \\
            Government (Westminister) budget constraint \\
            Government (Holyrood) tax revenue \\
            Government (Westminister) tax revenue \\
            Resource (Scotland) constraint \\
            Resource (rUK) constraint \\
        }
    \end{tabular}
\end{table}
\begin{align}
    &\max\limits_{C_t, N_t, B_t/B^*_t} \E_t \left\{\sum^{\infty}_{t=0} \  \mathcal{U}(C_t, N_t)\right\} \\
&s.t. \nonumber
\end{align}
\renewcommand{\arraystretch}{8}
\begin{table}[H]
    \centering
    \begin{tabular}{l|l c}
    \makecell{Scenario 1 \\ (G: 2, $\tau_n: 0$)} &  \makecell{S HBC: \\ rUK HBC: \\ S GBC: \\ rUK GBC: \\ S TR: \\ rUK TR: \\ S RC: \\ rUK RC:} & 
        \makecell{
            $P_t C_t + \E_t [{R^{-1}_{t+1}}B_{t+1}] = B_t + W_t N_t + T_t$ \\
            $P^*_t C^*_t + \E_t [{R^{*-1}_{t+1}}B^*_{t+1}] = B^*_t + W^*_t N^*_t + T^*_t$ \\
            $\E_t [{R^{-1}_{t+1}}B_{t+1}] + T_t = P_tG_t + B_t$\\
            $\E_t [{R^{*-1}_{t+1}}B^*_{t+1}] + T^*_t = P^*_tG^*_t + B^*_t$ \\
            $T_t = G_t$ \\
            $T^*_t = G^*_t$ \\
            $Y_t = C_t + G_t$ \\
            $Y^*_t = C^*_t + G^*_t$
        }  \\ 
    \makecell{Scenario 2 \\ (G: 1, $\tau_n: 0$)} &  \makecell{S HBC: \\ rUK HBC: \\ S GBC: \\ rUK GBC: \\ S TR: \\ rUK TR: \\ S RC: \\ rUK RC:} & 
        \makecell{
            $P_t C_t + \E_t [{R^{*-1}_{t+1}}B^*_{t+1}] = B^*_t + W_t N_t + \varpi T^*_t$ \\
            $P^*_t C^*_t + \E_t [{R^{*-1}_{t+1}}B^*_{t+1}] = B^*_t + W^*_t N^*_t + (1-\varpi)T^*_t$ \\
            N/A\\
            $\E_t [{R^{*-1}_{t+1}}B^*_{t+1}] + T^*_t = P^*_tG^*_t + B^*_t$\\
            N/A \\
            $T^*_t = P^*_tG^*_t$ \\
            $Y_t = C_t + \varpi G^*_t$ \\
            $Y^*_t = C^*_t + (1-\varpi) G^*_t$
        }   \\ 
    \makecell{Scenario 3 \\ (G: 2, $\tau_n: 1$)} &  \makecell{S HBC: \\ rUK HBC: \\ S GBC: \\ rUK GBC: \\ S TR: \\ rUK TR: \\ S RC: \\ rUK RC:} & 
    \makecell{
        $P_t C_t + \E_t [{R^{-1}_{t+1}}B_{t+1}] = B_t + (1-\tau_n)W_t N_t + T_t$ \\
        $P^*_t C^*_t + \E_t [{R^{*-1}_{t+1}}B^*_{t+1}] = B^*_t + (1-\tau_n)W^*_t N^*_t + T^*_t$ \\
        $\E_t [{R^{-1}_{t+1}}B_{t+1}] + T_t = P_tG_t + B_t$\\ 
        $\E_t [{R^{*-1}_{t+1}}B^*_{t+1}] + T^*_t = P^*_tG^*_t + B^*_t$\\ 
        $T_t = \tau_n W_t N_t$ \\
        $T^*_t = \tau_n W^*_t N^*_t$ \\
        $Y_t = C_t + G_t$ \\
        $Y^*_t = C^*_t + G^*_t$
    }  \\
    \makecell{Scenario 4 \\ (G: 1, $\tau_n: 1$)} &  \makecell{S HBC: \\ rUK HBC: \\ S GBC: \\ rUK GBC: \\ S TR: \\ rUK TR: \\ S RC: \\ rUK RC:} & 
    \makecell{
        $P_t C_t + \E_t [{R^{*-1}_{t+1}}B^*_{t+1}] = B^*_t + (1-\tau_n)W_t N_t + \varpi T_t$ \\
        $P^*_t C^*_t + \E_t [{R^{*-1}_{t+1}}B^*_{t+1}] = B^*_t + (1-\tau_n)W^*_t N^*_t + (1-\varpi)T^*_t$ \\
        N/A \\
        $\E_t [{R^{*-1}_{t+1}}B^*_{t+1}] + T^*_t = P^*_tG^*_t + B^*_t$\\ 
        N/A \\
        $T^*_t = \varpi\tau_n W_t N_t + (1-\varpi)\tau_n W^*_t N^*_t$ \\
        $Y_t = C_t + \varpi G^*_t$ \\
        $Y^*_t = C^*_t + (1-\varpi) G^*_t$
    }  
    \end{tabular}
    \end{table}
