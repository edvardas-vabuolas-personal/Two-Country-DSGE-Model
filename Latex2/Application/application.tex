In terms of RBC family of models, application normally refers to the process of calibrating model parameters and analysing impulse response functions, given temporary shocks in interest rate, level of technology, world output, etc. In 20XX, Smith and Watson (\textbf{??}) applied Markov Chain Monte Carlo (MCMC) technique to estimate model parameters, which has become a standard practice. However, due to the limitations of the word count and the scope of this dissertation, the specifics of Bayesian estimation will not be delved into. Blake and Mumtaz (2017) handbook of applied Bayesian econometrics has a dedicate section for implementation of random walk Metropolis-Hastings algorithm for DSGE models and even includes complementary Matlab code repository. For model parameters estimation and impulse response functions, this dissertation will utilise out-of-the-box solutions offered by Dynare (\textbf{refs}). This section will cover parameters, data, and estimation results. The next section will cover impulse response functions.

Whilst estimating all (or most) of the parameters is an option, the majority of parameter values will be calibrated in line with the literature. The reason behind the decision is the simplicity of our model. The model does not consider capital, investment, sticky wages, limited access to financial instruments, and many other extensions in the literature that are found to improve data fit. The dissertation, however, does estimate three sensitive parameters key to the research question: $\phi_b$, $\phi_g$, and $\rho_g$. The table below lists all model parameters and reasoning behind their associated values:

\begin{table}[H]
    \centering
    \begin{tabular}{ll|lll}
        Parameter & Value & Parameter & Value & Strategy \\
        \hline
        \hline
        $\beta$ & 0.99 & $\beta^*$ & $\beta$ & Calculated so that $R_t = 1.01$\\
        $\sigma$ & 1.00 & $\sigma^*$ & $\sigma$ & (\textbf{refs})\\
        $\varphi$ & 6.00 & $\varphi^*$ & $\varphi$ & (\textbf{refs})\\
        $\alpha$ & $\frac{1}{3}$ & $\alpha^*$ & $\alpha$ & (\textbf{refs})\\
        $\epsilon$ & 6.00 & $\epsilon^*$ & $\epsilon$ & (\textbf{refs})\\
        $\theta$ & $\frac{3}{4}$ & $\theta^*$ & $\theta$ & (\textbf{refs})\\
        $\upsilon$ & 0.60 & $\upsilon^*$ & $\upsilon$ & (\textbf{refs})\\
        $\rho_a$ & 0.50 & $\rho_a^*$ & $\rho_a$ & (\textbf{refs})\\
        $\eta$ & 1 & $\eta^*$ & $\eta$ & (\textbf{refs})\\
        $\rho_z$ & 0.50 & $\rho_z^*$ & $\rho_z$ & (\textbf{refs})\\
        $\rho_\nu$ & 0.50 & $\rho^*_\nu$ & $\rho_\nu$ & (\textbf{refs})\\
        $\phi_\pi$ & 1.50 & $\phi^*_\pi$ & $\phi_\pi$ & (\textbf{refs})\\
        $\phi_y$ & 0.125 & $\phi^*_y$ & $\phi_y$ & (\textbf{refs})\\
        $G_Y$ & 0.25 & $G_Y^*$ & 0.2 & (\textbf{refs})\\
        $C_Y$ & 1.00 - $G_Y$ & $C_Y^*$ & 1 - $G_Y^*$ & Implied by $1=\frac{C}{Y} + \frac{G}{Y}$\\
        $Y_C$ & $1/{C_Y}$ & $Y_C^*$ & $1/{C_Y^*}$ & Implied \\
        $G_C$ & $Y_C$ - 1 & $G_C^*$ & $Y_C^*$ - 1 & Implied \\
        \multicolumn{5}{l}{\textit{For multiple governments:}} \\
        $\rho_g$ & 0.90 & $\rho_g^*$ & $\rho_g$ & Estimated and Gali, Lopez, Valles, 2005\\
        $\phi_b$ & 0.33 & $\phi_b^*$ & $\phi_b$ & Estimated and Gali, Lopez, Valles, 2005\\
        $\phi_g$ & 0.10 & $\phi_g^*$ & $\phi_g$ & Estimated and Gali, Lopez, Valles, 2005\\
        $\tau$ & 0.20 & $\tau^*$ & $\tau$ & (\textbf{refs})\\
        \multicolumn{5}{l}{\textit{For a single government:}} \\
         &&$\rho_g^{UK}$  & 0.90  & Estimated and Gali, Lopez, Valles, 2005\\
         &&$\phi_b^{UK}$  & 0.33  & Estimated and Gali, Lopez, Valles, 2005\\
         &&$\phi_g^{UK}$  & 0.10  & Estimated and Gali, Lopez, Valles, 2005\\
         &&$\tau^{UK}$    & 0.20  & (\textbf{refs})\\
    \end{tabular}
    \caption{Model parameters, their values, and the strategy for obtaining the values}
\end{table}


