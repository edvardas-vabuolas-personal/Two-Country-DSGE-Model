Regarding the RBC family of models, the application typically refers to calibrating model parameters and analysing impulse response functions, given temporary shocks in interest rate, level of technology, world output, etc. In 20XX, Smith and Watson (\textbf{??}) applied Markov Chain Monte Carlo (MCMC) technique to estimate model parameters, which has become standard practice. However, due to the limitations of the word count and the scope of this dissertation, the specifics of Bayesian estimation will not be delved into. Blake and Mumtaz's (2017) handbook of applied Bayesian econometrics has a dedicated section for implementing random walk Metropolis-Hastings algorithm for DSGE models and even includes a complementary Matlab code repository. For model parameters estimation and impulse response functions, this dissertation will utilise out-of-the-box solutions offered by the Dynare computer package (\textbf{refs}). This section will cover parameters, data, and estimation results. The following section will provide an analysis of impulse response functions under different policy scenarios and a parameter sensitivity analysis.

While estimating all parameters is an option, in theory, most parameter values will be calibrated in line with the literature. The reason behind the decision is the simplicity of our model. The model does not consider capital, habit formation, sticky wages, limited access to financial instruments, and many other extensions in the literature that improve data fit. This makes our data uninformative for some parameters, and estimation results heavily rely on priors. Table \ref{table:model_parameters} below lists calibrated (as opposed to estimated) model parameters for Scotland and the rest of the UK. In the majority of the cases, parameters are assumed to be symmetrical. The asymmetry in responses to government spending is assumed to stem from different steady state ratios and estimated fiscal policy feedback parameters $\phi_b$ and $\phi_g$, as well as government spending persistence parameter $\rho_g$. The last column provides the reasoning behind all model parameter values.
\import{./Application/Tables}{model_parameters.tex}
The dissertation, however, does estimate three sensitive parameters key to the research question: $\phi_b$, $\phi_g$, and $\rho_g$ (and their rUK / UK counterparts). The acceptance rate per chain was c. 22\%, close to the optimal acceptance rate suggested by \textcite{roberts_2001_optimal}. The acceptance rate in the range of 23\% ensures that the variance of candidate parameter values is neither too large nor too small (ibid), meaning that all sets of parameters had a reasonable probability of being drawn. The number of MH draws and initial (``burn-in'') draws was set to 300.000 and 100.000, respectively. Figures (\textbf{??})-(\textbf{??}) display how the 300.000 draws varied. The variance of the draws suggests that the serial correlation of lagged draws was not persistent (fading), which is also indicative of successful estimation Roberts and Rosenthal (2001). The posterior means of all estimated parameters are close to their prior means, which can be primarily attributed to tight prior variance. However, the two means do not coincide, suggesting (to some degree) an informative estimation of the parameters. In fact, estimation of $\phi_b$ and $\phi_g$ are bound to be difficult given that the fiscal rule, while being reasonable and intuitive, is imposed arbitrarily to ``close'' the model. The true rule followed by the decision-makers, however, might be more complex and considerate of other metrics not captured by the model. All parameter values were drawn from Beta distribution due to it's values being bounded by zero and one. Table \ref{tab:priors_and_posteriors} lists distributions, prior means and variances, and posterior means of all estimated parameters. 
\import{./Application/Tables}{priors_posteriors.tex}
\import{./Graphs}{MCMC_Scot.tex}
Appendix C displays draws of corresponding variables for the rest of the UK and UK. In terms of data, most time series for Scotland and the rest of the UK were acquired from the Quarterly National Accounts of Scotland (QNAS) and the Quarterly Nationals Account (QNA), respectively. In contrast to \textcite{ricci_2019_essays}, who used 1998Q1-2007Q4 dataset, this dissertation utilised full QNA/QNAS dataset (1998Q1-2022Q4) for the variables of interest.\enlargethispage{\baselineskip}\footnote{\textcite{ricci_2019_essays} argued that their DSGE model is not ``equipped to account for''\parencite[123]{ricci_2019_essays} the global financial crisis of 2008. However, the use of a full dataset was found to improve estimation results for this model and is a more transparent choice. Therfore, the dissertation opted for 1998Q1-2022Q4 sample date range.\nopagebreak}
\import{./Application/Tables}{sources.tex}
Each time series were then processed in line with Pfifer (2019). That is, the time series for Scotland were adjusted to 2018 price levels and divided by the number of (estimated) working-age residents in Scotland. Then, the time series were expressed in natural logarithms and first-differenced to induce stationarity. Finally, the series was demeaned. Identical data processing steps were taken for the rest of the UK time series, except they were adjusted to 2023 price levels. Note that the use of 2018 and 2023 deflators does not have an impact on variables' sample cross-moments but allows to be in line with QNA/QNAS. Also note, that data was not available for all regions of the UK. Therefore, in most cases, the rest of the UK time series were calculated by taking Scotland's time series and subtracting them from the UK-wide ones. The resulting values should be approximately equal to the rest of the UK ones, as long as QNAS and QNA use symmetrical accounting/data processing methods for each pair of time series. Appendix B displays plots of the pre- and post-transformed time series.