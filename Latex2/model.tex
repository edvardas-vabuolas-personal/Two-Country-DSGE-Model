% \section*{Two-country DSGE model for Scotland and the rest of the UK}

% Closely follows \textcite{jordigal_2015_monetary} and \textcite{gali_2005_monetary}
% \begin{enumerate}
%     \item Home country: Scotland (notation - $X$)
%     \item Foreign country: rUK (notation - $X^*$)
%     \item Notation for UK-wide variables - $X^{UK}$
%     \item  Scotland and rUK are SOEs
%           \begin{enumerate}
%               \item They trade, take world output, inflation, and consumption as given and cannot influence it
%           \end{enumerate}
%     \item Calvo staggered prices, no capital/investment
%     \item Scotland and RUK are assumed to be symmetrical in market structure and preferences
%     \item Monetary Union: There is a population-weighted UK-wide interest rate in place, and all four nations within the UK purchase government-issued bonds at this rate
%     \begin{enumerate}
%         \item We also consider counterfactual scenarios, i.e. where both Holyrood and Westminister can issue bonds at country-specific interest rate
%     \end{enumerate}
%     \item The government spending can be financed via a lump-sum tax, a labour (income) tax, and borrowing.
%     \item We consider 4 scenarios:
%     \begin{enumerate}
%         \item Two governments funded by lump-sum tax, \underline{two} of which can issue bonds
%         \item Two governments funded by lump-sum tax, \underline{one} of which can issue bonds
%         \item Two governments funded by an income tax, \underline{two} of which can issue bonds
%         \item Two governments funded by an income tax, \underline{one} of which can issue bonds
%     \end{enumerate}
% \end{enumerate}