\section*{Two-country DSGE model for Scotland and the rest of the UK}

My model. Closely follows \textcite{jordigal_2015_monetary} and \textcite{gali_2005_monetary}
\begin{enumerate}
    \item Home country: \textit{Scotland}
    \item  Foreign country: \textit{rUK}
    \item  Scotland and rUK are SOEs
          \begin{enumerate}
              \item They trade, take world output, inflation, and consumption as given and cannot influence it
          \end{enumerate}
    \item Calvo staggered prices, no capital/investment
    \item Scotland and RUK are assumed to be symmetrical in market structure and preferences
    \item Monetary Union: rUK sets the interest rate based on a weighted inflation and weighted deviation from the steady state;
    \item Trade linkage: output gap in Scotland ``today" heavily depends on the expected output gap in rUK ``tomorrow"
    \item Significant price pass-through: inflation in Scotland ``today'' heavily depends on the expected inflation in rUK ``tomorrow''; also nominal exchange rate equal to zero
\end{enumerate}

\textcite{ricci_2019_essays} two-country DSGE model for Scotland and the rest of the UK
\begin{enumerate}
    \item Home country: \textit{Scotland}
    \item  Foreign country: \textit{rUK}
    \item Two governments: Westminister \& Holyrood
    \item Barnett formula
    \item Oil sector
    \item Calvo staggered prices and wages, considers capital and investment
    \item Does not consider migration (they say it would make the model too complicated)
    \item They assume that Scotland and rUK jointly are a small open economy, whereas I model both economies as SOE
\end{enumerate}
\pagebreak
Key equations. Non-asterisk terms refer to Scotland, and asterisk terms refer to the rest of the UK. \\

NK Phillips Curves:
\begin{align*}
    \pi_t &= \chi_{\pi}\E_t\{ \pi_{t+1} \} + (1-\chi_{\pi})\E_t\{ \pi^*_{t+1} \} + \kappa \tilde{y}_t \\
    \pi^*_t &= \chi^*_{\pi}\E_t\{ \pi^*_{t+1} \} + (1-\chi^*_{\pi})\E_t\{ \pi_{t+1} \} + \kappa^* \tilde{y}^*_t
\end{align*}

Dynamic IS Curves:
\begin{align*}
    \tilde{y}_t &= \chi_{y}\E_t \{ \tilde{y}_{t+1} \} + (1-\chi_{y}) \E_t \{ \tilde{y}^*_{t+1} \} - {\sigma}^{-1}(i^*_t-(\chi_{\pi}\E_t\{ \pi_{t+1} \} + (1-\chi_{\pi})\E_t\{ \pi^*_{t+1} \}) - i_t^{nat}) \\
    \tilde{y}^*_t &= \chi^*_{y}\E_t \{ \tilde{y}^*_{t+1} \} + (1-\chi^*_{y}) \E_t \{ \tilde{y}_{t+1} \} - {\sigma^*}^{-1}(i^*_t-(\chi^*_{\pi}\E_t\{ \pi^*_{t+1} \} + (1-\chi^*_{\pi})\E_t\{ \pi_{t+1} \}) - i_t^{*nat})
\end{align*}

Monetary Policy Rule
\begin{align*}
    i^*_t &= \rho_i i^*_{t-1} + (1-\rho_i)(\phi_{\pi}(\varpi\pi_t + (1-\varpi)\pi^*_t) + \phi_{y}(\varpi \tilde{y}_t + (1-\varpi)\tilde{y}^*_t)) + \epsilon_{i,t}
\end{align*}

Estimated parameters
\begin{align*}
    \chi_{\pi} \sim \text{Beta}(\varpi, 0.03) && \chi^*_{\pi} \sim \text{Beta}((1-\varpi), 0.03)\\
    \chi_{y} \sim \text{Beta}(\varpi, 0.03) && \chi^*_{y} \sim \text{Beta}((1-\varpi), 0.03)
\end{align*}

Weights are given by:
\begin{equation*}
    \varpi = \frac{\sum_{t=1998Q1}^{2007Q4} \{\text{Population aged 14-65 in Scotland}\}_t}{\sum_{t=1998Q1}^{2007Q4} \{\text{Population aged 14-65 in the UK}\}_t} = 0.0862
\end{equation*}

\textbf{OR}

Estimated parameters
\begin{align*}
    \chi_{\pi} \sim \text{Beta}(\varpi, 0.03) && \chi^*_{\pi} \sim \text{Beta}(\varpi^*, 0.03)\\
    \chi_{y} \sim \text{Beta}(\varpi, 0.03) && \chi^*_{y} \sim \text{Beta}(\varpi^*, 0.03)
\end{align*}

Weights are given by:
\begin{align*}
    \varpi = \frac{1}{T}\sum\limits_{t=1998Q1}^{T=2007Q4}\text{Scotland's import from rUK} \\
    \varpi^* = \frac{1}{T}\sum\limits_{t=1998Q1}^{T=2007Q4}\text{rUK's import from Scotland}
\end{align*}

\pagebreak
``No. of governments'': Number of governments that can issue bonds and accrue debt to fund public services. In all scenarios, government spending is assumed to be an AR(1) exogenous process.
\begin{enumerate}
    \item Scenario 1
    \begin{enumerate}
        \item No. of governments: 1
        \item Public expenses funded by (wasteful) \underline{lump-sum tax}
    \end{enumerate}
    \item Scenario 2
    \begin{enumerate}
        \item No. of governments: 2
        \item Public expenses funded by (wasteful) \underline{lump-sum tax}
    \end{enumerate}
    \item Scenario 3
    \begin{enumerate}
        \item No. of governments: 1
        \item Public expenses funded by (distortionary) \underline{income tax}
    \end{enumerate}
    \item Scenario 4
    \begin{enumerate}
        \item No. of governments: 2
        \item Public expenses funded by (distortionary) \underline{income tax}
    \end{enumerate}
\end{enumerate}
\pagebreak
\begin{table}[H]
    \centering
    \begin{tabular}{c|c}
        \makecell{
            S HBC \\
            rUK HBC \\
            S GBC \\
            rUK GBC \\
            S TR \\
            rUK TR \\
            S RC \\
            rUK RC \\
        } & 
        \makecell{
            Household (Scotland) budget constraint \\
            Household (rUK) budget constraint \\
            Government (Holyrood) budget constraint \\
            Government (Westminister) budget constraint \\
            Government (Holyrood) tax revenue \\
            Government (Westminister) tax revenue \\
            Resource (Scotland) constraint \\
            Resource (rUK) constraint \\
        }
    \end{tabular}
\end{table}
\begin{align}
    &\max\limits_{C_t, N_t, B_t/B^*_t} \E_t \left\{\sum^{\infty}_{t=0} \  \mathcal{U}(C_t, N_t)\right\} \\
&s.t. \nonumber
\end{align}
\renewcommand{\arraystretch}{8}
\begin{table}[H]
    \centering
    \begin{tabular}{l|l c}
    \makecell{Scenario 1 \\ (G: 2, $\tau_n: 0$)} &  \makecell{S HBC: \\ rUK HBC: \\ S GBC: \\ rUK GBC: \\ S TR: \\ rUK TR: \\ S RC: \\ rUK RC:} & 
        \makecell{
            $P_t C_t + \E_t [{R^{-1}_{t+1}}B_{t+1}] = B_t + W_t N_t + T_t$ \\
            $P^*_t C^*_t + \E_t [{R^{*-1}_{t+1}}B^*_{t+1}] = B^*_t + W^*_t N^*_t + T^*_t$ \\
            $\E_t [{R^{-1}_{t+1}}B_{t+1}] + T_t = P_tG_t + B_t$\\
            $\E_t [{R^{*-1}_{t+1}}B^*_{t+1}] + T^*_t = P^*_tG^*_t + B^*_t$ \\
            $T_t = G_t$ \\
            $T^*_t = G^*_t$ \\
            $Y_t = C_t + G_t$ \\
            $Y^*_t = C^*_t + G^*_t$
        }  \\ 
    \makecell{Scenario 2 \\ (G: 1, $\tau_n: 0$)} &  \makecell{S HBC: \\ rUK HBC: \\ S GBC: \\ rUK GBC: \\ S TR: \\ rUK TR: \\ S RC: \\ rUK RC:} & 
        \makecell{
            $P_t C_t + \E_t [{R^{*-1}_{t+1}}B^*_{t+1}] = B^*_t + W_t N_t + \varpi T^*_t$ \\
            $P^*_t C^*_t + \E_t [{R^{*-1}_{t+1}}B^*_{t+1}] = B^*_t + W^*_t N^*_t + (1-\varpi)T^*_t$ \\
            N/A\\
            $\E_t [{R^{*-1}_{t+1}}B^*_{t+1}] + T^*_t = P^*_tG^*_t + B^*_t$\\
            N/A \\
            $T^*_t = P^*_tG^*_t$ \\
            $Y_t = C_t + \varpi G^*_t$ \\
            $Y^*_t = C^*_t + (1-\varpi) G^*_t$
        }   \\ 
    \makecell{Scenario 3 \\ (G: 2, $\tau_n: 1$)} &  \makecell{S HBC: \\ rUK HBC: \\ S GBC: \\ rUK GBC: \\ S TR: \\ rUK TR: \\ S RC: \\ rUK RC:} & 
    \makecell{
        $P_t C_t + \E_t [{R^{-1}_{t+1}}B_{t+1}] = B_t + (1-\tau_n)W_t N_t + T_t$ \\
        $P^*_t C^*_t + \E_t [{R^{*-1}_{t+1}}B^*_{t+1}] = B^*_t + (1-\tau_n)W^*_t N^*_t + T^*_t$ \\
        $\E_t [{R^{-1}_{t+1}}B_{t+1}] + T_t = P_tG_t + B_t$\\ 
        $\E_t [{R^{*-1}_{t+1}}B^*_{t+1}] + T^*_t = P^*_tG^*_t + B^*_t$\\ 
        $T_t = \tau_n W_t N_t$ \\
        $T^*_t = \tau_n W^*_t N^*_t$ \\
        $Y_t = C_t + G_t$ \\
        $Y^*_t = C^*_t + G^*_t$
    }  \\
    \makecell{Scenario 4 \\ (G: 1, $\tau_n: 1$)} &  \makecell{S HBC: \\ rUK HBC: \\ S GBC: \\ rUK GBC: \\ S TR: \\ rUK TR: \\ S RC: \\ rUK RC:} & 
    \makecell{
        $P_t C_t + \E_t [{R^{*-1}_{t+1}}B^*_{t+1}] = B^*_t + (1-\tau_n)W_t N_t + \varpi T_t$ \\
        $P^*_t C^*_t + \E_t [{R^{*-1}_{t+1}}B^*_{t+1}] = B^*_t + (1-\tau_n)W^*_t N^*_t + (1-\varpi)T^*_t$ \\
        N/A \\
        $\E_t [{R^{*-1}_{t+1}}B^*_{t+1}] + T^*_t = P^*_tG^*_t + B^*_t$\\ 
        N/A \\
        $T^*_t = \varpi\tau_n W_t N_t + (1-\varpi)\tau_n W^*_t N^*_t$ \\
        $Y_t = C_t + \varpi G^*_t$ \\
        $Y^*_t = C^*_t + (1-\varpi) G^*_t$
    }  
    \end{tabular}
    \end{table}
\pagebreak
\tiny
\begin{align}
    T^*_t &= \varpi\tau_t W_t N_t + (1-\varpi)\tau_t W^*_t N^*_t \\
    T^* &= \varpi\tau W N + (1-\varpi)\tau W^* N^* \label{eq:tr_ss} \quad \text{(Steady state)}\\
    &\text{Using Uhlig's (1999) method, $X_tY_t \approx XY\mathbf{e}^{\tilde{X}_t + \tilde{Y}_t}$: }\\
    T\mathbf{e}^{\tilde{T}^*_t} &= \varpi \tau W N \mathbf{e}^{\tilde{\tau}_t+\tilde{W}_t+\tilde{N}_t} + (1-\varpi) \tau W^* N^* \mathbf{e}^{\tilde{\tau}_t+\tilde{W}^*_t+\tilde{N}^*_T} \\
    &\text{Using $\mathbf{e}^{\tilde{X}_t} \approx 1 + {\tilde{X}_t}$: }\\
    T^* (1 + {\tilde{T}^*_t}) &= \varpi \tau W N (1 + {\tilde{\tau}_t+\tilde{W}_t+\tilde{N}_t}) + (1-\varpi) \tau W^* N^* (1 + {\tilde{\tau}_t+\tilde{W}^*_t+\tilde{N}^*_T}) \\
    &\text{Subtract (\ref{eq:tr_ss}): } \\
    T^* (1 + {\tilde{T}^*_t}) - T^* &= \varpi \tau W N (1 + {\tilde{\tau}_t+\tilde{W}_t+\tilde{N}_t}) - \varpi\tau W N + (1-\varpi) \tau W^* N^* (1 + {\tilde{\tau}_t+\tilde{W}^*_t+\tilde{N}^*_T}) - (1-\varpi)\tau W^* N^* \\
    T^*[(1 + {\tilde{T}^*_t}) - 1] &= \varpi \tau W N [(1 + {\tilde{\tau}_t+\tilde{W}_t+\tilde{N}_t}) - 1] + (1-\varpi) \tau W^* N^* [(1 + {\tilde{\tau}_t+\tilde{W}^*_t+\tilde{N}^*_T}) - 1] \\
    T^*\tilde{T}^*_t &= \varpi \tau W N (\tilde{\tau}_t + \tilde{W}_t + \tilde{N}_t) + (1-\varpi) \tau W^* N^* (\tilde{\tau}_t+\tilde{W}^*_t+\tilde{N}^*_T) \\
    &\text{Divide by $T^*$}: \\
    \tilde{T}^*_t &= \frac{\tau W N}{T^*}\times \varpi (\tilde{\tau}_t + \tilde{W}_t + \tilde{N}_t) + \frac{\tau W^* N^*}{T^*} \times (1-\varpi) (\tilde{\tau}_t+\tilde{W}^*_t+\tilde{N}^*_t)
\end{align}

\begin{align}
    C_t^\sigma N_t^\varphi &= \frac{W_t}{P_t}(1-\tau_t) \\
    C_t^\sigma N_t^\varphi &= \frac{W_t}{P_t} - \frac{W_t}{P_t}\tau_t \\
    C^\sigma N^\varphi &= \frac{W}{P} - \frac{W}{P}\tau \label{eq:real_wage_ss} \quad \text{(Steady state)}\\
    &\text{Using Uhlig's (1999) method, $X_tY_t \approx XY\mathbf{e}^{\tilde{X}_t + \tilde{Y}_t}$: }\\
    C^\sigma N^\varphi\mathbf{e}^{\sigma c_t + \varphi n_t} &= \frac{W}{P}\mathbf{e}^{w_t - p_t} - \frac{W}{P}\tau \mathbf{e}^{w_t - p_t + \tilde{\tau}_t}\\
    &\text{Using $\mathbf{e}^{\tilde{X}_t} \approx 1 + {\tilde{X}_t}$: }\\
    C^\sigma N^\varphi({1+\sigma c_t + \varphi n_t}) &= \frac{W}{P}({1+ w_t - p_t}) - \frac{W}{P}\tau ({1+ w_t - p_t + \tilde{\tau}_t})\\
    &\text{Subtract (\ref{eq:real_wage_ss}): } \\
    C^\sigma N^\varphi({\sigma c_t + \varphi n_t}) &= \frac{W}{P}({w_t - p_t}) - \frac{W}{P}\tau ({w_t - p_t + \tilde{\tau}_t})\\
    C^\sigma N^\varphi({\sigma c_t + \varphi n_t}) &= \frac{W}{P}\left[({w_t - p_t}) - \tau ({w_t - p_t + \tilde{\tau}_t})\right]\\
    C^\sigma N^\varphi({\sigma c_t + \varphi n_t}) &= \frac{W}{P}\left[({w_t - p_t}) - \tau ({w_t - p_t}) - \tau\tilde{\tau}_t\right]\\
    C^\sigma N^\varphi({\sigma c_t + \varphi n_t}) &= \frac{W}{P}\left[(1-\tau)({w_t - p_t}) - \tau\tilde{\tau}_t\right]\\
    C^\sigma N^\varphi \frac{P}{W} ({\sigma c_t + \varphi n_t})  &= \left[(1-\tau)({w_t - p_t}) - \tau\tilde{\tau}_t\right]\\
    (1-\tau)({\sigma c_t + \varphi n_t})  &= \left[(1-\tau)({w_t - p_t}) - \tau\tilde{\tau}_t\right]\\
    {\sigma c_t + \varphi n_t}  &= {w_t - p_t} - \frac{\tau}{1-\tau}\tilde{\tau}_t\\
\end{align}