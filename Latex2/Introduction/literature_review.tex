\subsection{Literature Review}

On the 10th of May, 2023, the Monetary Policy Committee at the Bank of England gathered to discuss the latest international and domestic data on economic activity. Even though the Committee has a 2\% CPI target, the UK's economy had undergone a sequence of very large and unexpected shocks and disturbances, resulting in twelve-month CPI inflation above 10\%. The majority of the Committee members (78\%) believed that an increase in interest rate ``was warranted" \parencite[4]{boe_2023_monetary}, while the remaining members believed that the CPI inflation will ``fall sharply in 2023" \parencite[5]{boe_2023_monetary} as a result of the economy naturally adjusting to the effects of the energy price shocks. They feared that the preceding increases in the interest rate have not yet been internalised and raising the interest rate any further could result in a reduction of inflation ``well below the target" \parencite[5]{boe_2023_monetary}. This is an excellent illustration of the ``informal dimension of the monetary policy process", that \parencite[26]{gals_2007_macroeconomic} referred to in their work explaining modern macro models and new frameworks. According to them, while the informal dimension cannot be removed, we can build formal and rigorous models that would help the Committee and institutions-alike to understand ``objectives of the monetary policy and how the latter should be conducted in order to attain those objectives" \parencite[2]{jordigal_2015_monetary}. This task is not straightforward and has been central (albeit - fruitful) to most macroeconomic research in the past decades. The following section of the literature review will present a brief evolution of the study of business cycles and monetary theory. It will be followed by an overview of large macroeconomic models adopted by central banks and international organisations to illustrate the relevance of this research. The final part of the literature review will discuss \textcolor{red}{\{NIESR policy question, also Scotland-UK deltas\}} and the latest research on the issue.

\textcite{blanchard_2000_what} offers a compact description of macroeconomic research in the twentieth century. In their panegyric and optimistic essay, the researcher argues that the century can be divided into three epochs based on the prevailing beliefs about the economy and frameworks of the time: Pre-1940, From 1940 to 1980, and Post-1980. According to them, pre-1940 epoch was the epoch of exploration, with economists laying foundations for and being primarily concerned with the Monetary Theory (``Why does money affect output?'') and Business Cycles (``What are the major schocks that affect output?'') as two disconnected issues.

of that time did not differ drastically from what is believed today, e.g. short-run money non-neutrality and long-run neutrality, but the economic models were ``incomplete and partial equilibrium in nature'' \parencite[1377]{blanchard_2000_what}. While modern models introduce nominal rigidities and frictions to explain money non-neutrality, then it was believed that higher prices ``excited'' business

In the 1980s, \textcite{kydland_1982_time} and \textcite{prescott_1986_theory} published seminal papers on the Real Business Cycles (RBC) theory. According to \textcite[2]{jordigal_2015_monetary}, frameworks presented in the papers were ``provided the main reference'' and 