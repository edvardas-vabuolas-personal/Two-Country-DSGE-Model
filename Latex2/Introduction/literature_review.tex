\subsection{Literature Review}
On the 10th of May, 2023, the Monetary Policy Committee at the Bank of England gathered to discuss the latest international and domestic data on economic activity. Even though the Committee has a 2\% CPI target, the UK's economy had undergone a sequence of very large and unexpected shocks and disturbances, resulting in twelve-month CPI inflation above 10\%. The majority of the Committee members (78\%) believed that an increase in interest rate ``was warranted" \parencite[4]{boe_2023_monetary}, while the remaining members believed that the CPI inflation will ``fall sharply in 2023" \parencite[5]{boe_2023_monetary} as a result of the economy naturally adjusting to the effects of the energy price shocks. They feared that the preceding increases in the interest rate have not yet been internalised and raising the interest rate any further could result in a reduction of inflation ``well below the target" \parencite[5]{boe_2023_monetary}. This is an excellent illustration of the ``informal dimension of the monetary policy process", that \parencite[26]{gals_2007_macroeconomic} referred to in their work explaining modern macro models and new frameworks. According to them, while the informal dimension cannot be removed, we can build formal and rigorous models that would help the Committee and institutions-alike to understand ``objectives of the monetary policy and how the latter should be conducted in order to attain those objectives" \parencite[2]{jordigal_2015_monetary}. This task is not straightforward and has been central (albeit - fruitful) to most macroeconomic research in the past decades. The following section of the literature review will present a brief evolution of the study of business cycles and monetary theory. It will be followed by an overview of large macroeconomic models adopted by central banks and international organisations to illustrate the relevance of this research. The final part of the literature review will discuss \textcolor{red}{\{NIESR policy question, also something about Scotland and the UK deltas\}} and the latest research on the issue.

In the 1980s, Kydland and Prescott (1982) and Prescott (1986) published seminal papers on the Real Business Cycles (RBC) theory. The researchers built theoretical macroeconomic models to explain observed 