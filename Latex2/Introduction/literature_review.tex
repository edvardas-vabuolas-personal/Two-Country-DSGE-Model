\subsection{Literature Review}

On the 10th of May, 2023, the Monetary Policy Committee at the Bank of England gathered to discuss the latest international and domestic data on economic activity. Even though the Committee has a 2\% CPI target, the UK's economy had undergone a sequence of very large and unexpected shocks and disturbances, resulting in twelve-month CPI inflation above 10\%. The majority of the Committee members (78\%) believed that an increase in interest rate ``was warranted" \parencite[4]{boe_2023_monetary}, while the remaining members believed that the CPI inflation will ``fall sharply in 2023" \parencite[5]{boe_2023_monetary} as a result of the economy naturally adjusting to the effects of the energy price shocks. They feared that the preceding increases in the interest rate have not yet been internalised and raising the interest rate any further could result in a reduction of inflation ``well below the target" \parencite[5]{boe_2023_monetary}. This is an excellent illustration of the ``informal dimension of the monetary policy process", that \parencite[26]{gals_2007_macroeconomic} referred to in their work explaining modern macro models and new frameworks. According to them, while the informal dimension cannot be removed, we can build formal and rigorous models that would help the Committee and institutions-alike to understand ``objectives of the monetary policy and how the latter should be conducted in order to attain those objectives" \parencite[2]{jordigal_2015_monetary}. This task is not straightforward and has been central (albeit - fruitful) to most macroeconomic research in the past decades. The following section of the literature review will present a brief evolution of the study of business cycles and monetary theory. It will be followed by an overview of large macroeconomic models adopted by central banks and international organisations to illustrate the relevance of this research. The final part of the literature review will outline insitutional and fiscal policy differences in Scotland and the rest of the UK.

\textcite{blanchard_2000_what} offers a compact description of macroeconomic research in the twentieth century. In their panegyric and optimistic essay, the researcher argues that the century can be divided into three epochs based on the prevailing beliefs about the economy and frameworks of the time: Pre-1940, From 1940 to 1980, and Post-1980. 
\import{./Graphs}{timeline.tex}
According to them, the pre-1940 epoch was the epoch of exploration, with economists primarily concerned with the Monetary Theory (``Why does money affect output?'') and Business Cycles (``What are the major shocks that affect output?''). Even though both of those puzzles fall under the study of ``macroeconomics'', the term did not even appear in the literature until the mid-1940s \parencite{blanchard_2000_what}. The Monetary Theory ideas of that time did not differ drastically from what is believed today, e.g. short-run money non-neutrality and long-run neutrality, but the economic models were ``incomplete and partial equilibrium in nature'' \parencite[1377]{blanchard_2000_what}. The business cycles were attributed to ``real factors'', such as technological innovations \parencite{blanchard_2000_what}, and this belief persisted until more data became available and more sophisticated time series methods were applied in the early 2000s \parencite[3]{jordigal_2015_monetary}. The subsequent epoch was ``the golden age of macroeconomics'' \parencite[1379]{blanchard_2000_what}. \textcolor{red}{Hicks (1937)} formalised the \textit{IS-LM} framework \textcolor{red}{TBC}

In the 1980s, \textcite{kydland_1982_time} and \textcite{prescott_1986_theory} published seminal papers on the Real Business Cycles (RBC) theory. According to \textcite[2]{jordigal_2015_monetary}, frameworks presented in the papers ``provided the main reference'' and firmly established the use of dynamic stochastic general equilibrium (DSGE) models as crucial tools for macroeconomic analysis. The models allow quantitative analysis and incorporation of data either via calibration or estimation of parameters. \textcolor{red}{TBC}

International Monetary Fund (IMF)

\textbf{Government in DSGE literature review}

To begin with, RBC models predict a negative response in consumption following an increase in government spending. More specifically, government spending is modelled to absorb resources, which makes households worse off and incentivises more hours worked. Greater labour supply for any given wage reduces firms' marginal cost and induces output (Baxter and King, 1993: 319). That is, consumption, conditional on shocks in government spending, is countercyclical. Keynesian models, in stark contrast, predict the opposite. The countercyclicality is why the DSGE models sometimes do not consider government spending. 

Empirically, the findings of the Keynesian models are more in line with the observed macroeconomic patterns. For instance, Blanchard and Perotti (2002) performed a VAR analysis on the dynamics of consumption and government spending. They built six structural VAR models, one for each component of GDP: output, consumption, government spending, investment, export, and import. The two other variables were taxes and government spending/output\footnote{That is, if the GDP component of interest is government spending, then the second variable is output; in all other cases, the second variable is government spending.}. The key finding of the analysis is that government spending has a positive effect on consumption. 

Gali, Lopez-Salido, and Valles (2005) show that NK DSGE models can be ``recovered'' by assuming that households have limited access to financial markets/saving technologies or are poor (they consume all of their labour income). Households that smooth their consumption by saving are often regarded as Ricardian households, while those that do not - non-Ricardian or \textit{hand-to-mouth} households. Just like Gali, Lopez-Salido, and Valles (2005) did, some of the latest literature NK DSGE literature models both types of households explicitly\footnote{In fact, there exists literature with more than two types of households. For instance, a recent paper by Eskelinen (2021) models poor hand-to-mouth, wealthy hand-to-mouth, and non-hand-to-mouth households.} with their ratio determined by a time-invarying exogenous coefficient. Arguably, such modelling would allow an improved fit of data. However, the key focus of this dissertation is the four policy scenarios, all of which heavily depend on governments' abilities to issue bonds and borrow. Hand-to-mouth households do not borrow/save, rendering bonds purposeless. While modelling hand-to-mouth households is even easier than the Ricardian households, modelling both types of households would drastically increase the complexity of the model, given the two-country setting. The absence of hand-to-mouth households is discussed in the limitations section.