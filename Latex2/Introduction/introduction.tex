In May, the Bank of England's Monetary Policy Committee (MPC) decided to raise the interest rate in response to the exceptionally high inflation of the consumer price index. The committee members were presented with the latest data and thorough analyses conducted by experienced teams of economists. However, the decision was not unanimous: a few members expressed concerns that the contractionary monetary policy shock had not yet been fully internalised, and further rate increases might lead to inflation falling below the target level. Despite this dissent, the majority of the committee members voted in favour of raising the interest rate, considering the potential negative impact of high inflation on households and businesses to outweigh the risks. The meeting minutes effectively demonstrate the informal aspects of the decision-making process. Modern macroeconomic frameworks provide insights into policy objectives and how they can be achieved through fiscal and monetary measures. This dissertation aims to develop a two-country New Keynesian Dynamic Stochastic General Equilibrium (NK DSGE) model for Scotland and the rest of the UK. The model also considers the existence of the world economy and international trading with other countries. The objective of the dissertation is not to build the most factually accurate model tailored to Scotland or the rest of the UK but build a model that would allow an assessment of regional variations in government spending responses under a predefined set of policy scenarios. More specifically, the dissertation considers: a unified fiscal authority scenario in which there exists a single government able to issue bonds and raise taxes and; two autonomous fiscal authorities scenario in which both Scotland and the rest of the UK can issue bonds and raise taxes. Moreover, the dissertation allows fiscal authorities to raise tax revenue by lump-sum or distortionary labour tax. Across all policy scenarios, the model assumes that one monetary authority exists - the Bank of England, and both countries adhere to a UK-wide interest rate. Most model parameters are calibrated in line with the academic literature on DSGE models in the UK. A few model parameters related to government spending and taxes are estimated using Bayesian techniques. Finally, the estimated values are used to derive impulse response functions that are key to delivering the dissertation objectives.

A literature review is presented in the first section. It covers an overview of the evolution of economic frameworks from the 1940s to today and the application of DSGE models by central banks and international organisations. The literature review section is followed by the description of the model. It is split into four parts: households, the government, firms, and equilibrium or market clearing conditions. Given the high number of employed equations by DSGE models, most of the derivations and proofs were put in Appendix, but the accompanying interpretation and intuition are provided in the main text. The remaining two sections are Application and Results. The application section covers the calibration of the model, a brief introduction to the Bayesian estimation of the model parameters and estimation results. The application section also presents the data used in the estimation process. The results section analyses endogenous variables' dynamic responses to two types of shocks: monetary and fiscal. It discusses government spending when the economy is introduced with a distortionary labour tax, as well as the asymmetry in responses when Scotland is considered fiscally autonomous. Finally, the dissertation provides a parameter sensitivity analysis - a demonstration and a discussion on how the dynamic responses vary given different sets of parameters. The conclusion summarises all sections of a dissertation, overviews the results and discusses the limitations of the model.